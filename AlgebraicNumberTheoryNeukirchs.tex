\documentclass[UTF8]{ctexart}
\usepackage[utf8]{inputenc}
\usepackage{natbib}
\usepackage{graphicx}
\usepackage{hyperref}
\usepackage{amssymb}
\usepackage{amsmath}
\usepackage{tikz-cd}
\usepackage{geometry}
\geometry{a4paper, margin=1in}

\begin{document}

\tableofcontents

\title{《代数数论》习题解答}
\author{高旭-GG译}
\date{2015年6月22日}
\maketitle

\section{结构说明}
问题,解答形式。一个大问题的内部,若需证明一些中间结论,就引理,命题等做相对标号,在大问题内部形成一个完整的逻辑链。各个大问题解答独立。有名的引理,定理,注明定理名称,全局生效。

\subsection{当前问题和进度条}
\begin{itemize}
    \item[0.] 问题2 引理1的验证
    \item[1.] 5.4和5.7连贯
    \item[2.] 基本做完了
    \item[3.] 闵可夫斯基理论 章节题目,需要一些先验(符号,背景等)
\end{itemize}

\section{第一章:高斯整数}

\subsection{练习题}

\begin{enumerate}

\item[1] 
\textbf{问题:} 证明 \(\alpha \in \mathbb{Z}[i]\) 是单位当且仅当 \(N(\alpha) = 1\)。\\
\textbf{解答:} 设 \(\alpha = x + i y\),其中 \(x, y \in \mathbb{Z}\),则 \(N(\alpha) = x^2 + y^2 \in \mathbb{Z}\)。  
若 \(\alpha\) 是单位,则存在 \(\alpha^{-1}\),使得 \(N(\alpha) N(\alpha^{-1}) = N(1) = 1\),因此 \(N(\alpha) = 1\)。  
反之,若 \(N(\alpha) = 1\),则其共轭 \(\overline{\alpha} = x - i y\) 是其逆,因为 \(\alpha \overline{\alpha} = (x + i y)(x - i y) = x^2 + y^2 = 1\)。  
故 \(\alpha \in \mathbb{Z}[i]\) 是单位当且仅当 \(N(\alpha) = 1\)。

\item[2] 
\textbf{问题:} 在环 \(\mathbb{Z}[i]\) 中,证明若 \(\alpha \beta = \varepsilon \gamma^n\),其中 \(\alpha, \beta\) 互素,\(\varepsilon\) 是单位,则存在单位 \(\varepsilon', \varepsilon''\) 使得 \(\alpha = \varepsilon' \xi^n\) 且 \(\beta = \varepsilon'' \eta^n\)。\\
\textbf{解答:} 我们证明一个更一般的结果: 
不妨临时将其设为
\textbf{命题 A:在唯一分解整环 (UFD) 中,若 \(\alpha \beta = \varepsilon \gamma^n\),\(\alpha, \beta\) 互素,\(\varepsilon\) 是单位,则 \(\alpha = \varepsilon' \xi^n\) 且 \(\beta = \varepsilon'' \eta^n\),其中 \(\varepsilon', \varepsilon''\) 是单位。}  

证明:根据唯一分解性质,设 \(\alpha = \varepsilon_1 \pi_1^{l_1} \pi_2^{l_2} \cdots \pi_s^{l_s}\),\(\beta = \varepsilon_2 \pi_1^{m_1} \pi_2^{m_2} \cdots \pi_s^{m_s}\),\(\gamma = \varepsilon_3 \pi_1^{n_1} \pi_2^{n_2} \cdots \pi_s^{n_s}\)。由于 \((\alpha, \beta) = 1\),有 \(l_j m_j = 0\)(\(j = 1, 2, \dots, s\))。由 \(\alpha \beta = \varepsilon \gamma^n\),得 \(l_j + m_j = n n_j\)。因此,对于每个 \(j\),要么 \(l_j = n n_j\),要么 \(m_j = n n_j\)。由此得出结论。

% \item[3] 
% \textbf{问题:} 证明方程 \(x^2 + y^2 = z^2\)(其中 \(x, y, z > 0\) 且 \((x, y, z) = 1\))的整数解(即“毕达哥拉斯三元组”)都可以通过公式 \(x = u^2 - v^2, y = 2uv, z = u^2 + v^2\) 给出,其中 \(u, v \in \mathbb{Z}, u > v > 0, (u, v) = 1\),且 \(u, v\) 不全为奇数(允许 \(x\) 和 \(y\) 互换)。\\
% \textbf{解答:} 设 \(\alpha = x + i y\),则 \((x, y, z)\) 是毕达哥拉斯三元组意味着 \(N(\alpha) = z^2\)。可假设 \((\alpha, \overline{\alpha}) = 1\)。由练习 2,得 \(\alpha = \varepsilon \xi^2\),其中 \(\varepsilon\) 是单位。设 \(\xi = u + i v\),则结论成立。
\item[3] 
\textbf{问题:} 证明方程 \(x^2 + y^2 = z^2\)(其中 \(x, y, z > 0\) 且 \((x, y, z) = 1\))的整数解(即“毕达哥拉斯三元组”)都可以通过公式 \(x = u^2 - v^2, y = 2uv, z = u^2 + v^2\) 给出,其中 \(u, v \in \mathbb{Z}, u > v > 0, (u, v) = 1\),且 \(u, v\) 不全为奇数(允许 \(x\) 和 \(y\) 互换)。\\
\textbf{解答:} 设 \(\alpha = x + i y\),则 \((x, y, z)\) 是毕达哥拉斯三元组意味着 \(N(\alpha) = z^2\)。可假设 \((\alpha, \overline{\alpha}) = 1\)。由于唯一分解整环中存在如下命题:

\begin{quote}
在环 \(\mathbb{Z}[i]\) 中,若 \(\alpha \beta = \varepsilon \gamma^n\),其中 \(\alpha, \beta\) 互素,\(\varepsilon\) 是单位,则存在单位 \(\varepsilon', \varepsilon''\) 使得 \(\alpha = \varepsilon' \xi^n\) 且 \(\beta = \varepsilon'' \eta^n\)。\\
证明:根据唯一分解性质,设 \(\alpha = \varepsilon_1 \pi_1^{l_1} \pi_2^{l_2} \cdots \pi_s^{l_s}\),\(\beta = \varepsilon_2 \pi_1^{m_1} \pi_2^{m_2} \cdots \pi_s^{m_s}\),\(\gamma = \varepsilon_3 \pi_1^{n_1} \pi_2^{n_2} \cdots \pi_s^{n_s}\)。由于 \((\alpha, \beta) = 1\),有 \(l_j m_j = 0\)(\(j = 1, 2, \dots, s\))。由 \(\alpha \beta = \varepsilon \gamma^n\),得 \(l_j + m_j = n n_j\)。因此,对于每个 \(j\),要么 \(l_j = n n_j\),要么 \(m_j = n n_j\)。由此得出结论。
\end{quote}

应用到此,设 \(\alpha = x + i y\),\(\beta = \overline{\alpha} = x - i y\),\(\varepsilon = 1\),\(\gamma = z\),\(n = 2\),则 \(\alpha \beta = z^2\)。由于 \((\alpha, \overline{\alpha}) = 1\),由上述结论,得 \(\alpha = \varepsilon \xi^2\),其中 \(\varepsilon\) 是单位。设 \(\xi = u + i v\),则:
\[
\xi^2 = (u + i v)^2 = u^2 - v^2 + 2uvi,
\]
\[
\alpha = \varepsilon \xi^2 = \varepsilon (u^2 - v^2 + 2uvi).
\]
取 \(\varepsilon = 1\),则:
\[
x + i y = u^2 - v^2 + 2uvi \implies x = u^2 - v^2, \quad y = 2uv,
\]
\[
z^2 = x^2 + y^2 = (u^2 - v^2)^2 + (2uv)^2 = (u^2 + v^2)^2 \implies z = u^2 + v^2.
\]
故结论成立。

\item[4] 
\textbf{问题:} 证明环 \(\mathbb{Z}[i]\) 不能被排序。\\
\textbf{解答:} 首先回顾有序环的定义:  

\textbf{定义:} 一个有序环是带有全序 \(\leqslant\) 的环 \(R\),满足:  
 若 \(a \leqslant b\),则 \(a + c \leqslant b + c\);  
 若 \(0 \leqslant a\) 且 \(0 \leqslant b\),则 \(0 \leqslant ab\)。  
若 \(a \neq 0\),\(0 \leqslant a\) 则 \(a\) 为正,\(a \leqslant 0\) 则 \(a\) 为负,\(0\) 既不正也不负。  

\textbf{命题:} 在有序环中,对于每个元素 \(a\),恰好满足以下之一:
\(a\) 为正,\(-a\) 为正,或 \(a = 0\)。特别地,\(a\) 为负当且仅当 \(-a\) 为正。  
假设 \(\mathbb{Z}[i]\) 可被全序 \(\leqslant\) 排序。考虑 \(i\):若 \(i\) 为正,则 \(-1 = i^2\) 为正,从而 \(1 = (-1)^2\) 也为正,与命题矛盾。故 \(\mathbb{Z}[i]\) 不能被排序。

\item[5] 
\textbf{问题:} 证明对于每个有理整数 \(d > 1\),环 \(\mathbb{Z}[\sqrt{-d}]\) 的单位仅为 \(\pm 1\)。\\
\textbf{解答:} 元素 \(\alpha = x + y \sqrt{-d}\) 的范数为 \(N(\alpha) = x^2 + d y^2\)。由于环\(\mathbb{Z}[i]\)中,\(\alpha\) 是单位当且仅当 \(N(\alpha) = 1\)。因此,\(\alpha\) 是单位等价于 \((x, y)\) 是方程 \(x^2 + d y^2 = 1\) 的整数解。因 \(d > 1\),该方程的唯一整数解为 \((\pm 1, 0)\)。故单位仅为 \(\pm 1\)。 

\item[6] 
\textbf{问题:} 证明对于每个无平方因子的整数 \(d > 1\),环 \(\mathbb{Z}[\sqrt{d}]\) 有无穷多个单位。\\
\textbf{解答:} 单位是指 \(\mathbb{Z}[\sqrt{d}]\) 中范数为 \(\pm 1\) 的元素。设 \(\alpha = x + y \sqrt{d}\),其范数为 \(N(\alpha) = (x + y \sqrt{d})(x - y \sqrt{d}) = x^2 - d y^2\)。\(\alpha\) 是单位当且仅当 \(x^2 - d y^2 = \pm 1\)。因此,问题等价于证明佩尔方程 \(x^2 - d y^2 = \pm 1\) 有无穷多整数解。由于 \(d > 1\) 且无平方因子,\(\sqrt{d}\) 是无理数,我们通过证明 \(x^2 - d y^2 = 1\) 有无穷多解来完成(因为存在一个非平凡解即可生成无穷多解)。  

\textbf{命题:} 对于每个无平方因子的整数 \(d > 1\),方程 \(x^2 - d y^2 = 1\) 有无穷多整数解。  
此处使用狄利克雷逼近定理证明:  

\textbf{引理(狄利克雷逼近定理):} 对于无理数 \(\theta\),存在无穷多对整数 \((x, y)\)(其中 \(y > 0\))使得:
\[
\left| \theta - \frac{x}{y} \right| < \frac{1}{y^2}.
\]
证明:通过 Dirichlet 抽屉原理,对于每个正整数 \(N\),存在整数 \(x\) 和 \(y\) 使得 \(1 \leq y \leq N\) 且:
\[
|x - y \theta| \leq \frac{1}{N+1}.
\]
具体而言,令 \(\theta_y = y \theta - [y \theta]\) 对于 \(1 \leq y \leq N\)。若存在某个 \(y\) 使得 \(\theta_y \in (0, \frac{1}{N+1})\) 或 \(\theta_y \in [\frac{N}{N+1}, 1)\),则 \(|[y \theta] - y \theta| < \frac{1}{N+1}\) 或 \(|([y \theta] + 1) - y \theta| < \frac{1}{N+1}\)。否则,\(N\) 个数 \(\theta_y\) 分布在 \(N-1\) 个区间 \([\frac{1}{N+1}, \frac{2}{N+1}), \dots, [\frac{N-1}{N+1}, \frac{N}{N+1})\),故存在 \(1 \leq y_1 < y_2 \leq N\) 和 \(0 < k < N\) 使得 \(\theta_{y_1}, \theta_{y_2} \in [\frac{k}{N+1}, \frac{k+1}{N+1})\)。于是:
\[
\left| ([y_2 \theta] - [y_1 \theta]) - (y_2 - y_1) \theta \right| < \frac{1}{N+1}.
\]
因此可以直接得出结论:给定一对 \((x, y)\) 满足 \(|x - y \theta| \leq \frac{1}{N+1}\),可选择更大的 \(N'\) 使得 \(|x - y \theta| > \frac{1}{N'+1}\),从而得到另一对 \((x', y')\),重复此过程可得无穷多对满足 \(\left| \theta - \frac{x}{y} \right| < \frac{1}{y^2}\) 的整数对。

\textbf{推论1:} 对于每个无平方因子的整数 \(d > 1\),存在无穷多对整数 \((x, y)\)(其中 \(y > 0\))使得:
\[
\left| x^2 - d y^2 \right| < 1 + 2\sqrt{d}.
\]
证明:由引理(狄利克雷逼近定理),存在无穷多对 \((x, y)\)(\(y > 0\))满足 \(|x - y \sqrt{d}| < \frac{1}{y}\)。对于这些对,有:
\[
\begin{aligned}
|x + y \sqrt{d}| &= |x - y \sqrt{d} + 2 y \sqrt{d}| \\
&\leq |x - y \sqrt{d}| + 2 y \sqrt{d} \\
&< \frac{1}{y} + 2 y \sqrt{d},
\end{aligned}
\]
因此:
\[
\begin{aligned}
\left| x^2 - d y^2 \right| &= |x - y \sqrt{d}| |x + y \sqrt{d}| \\
&< \frac{1}{y} \left( \frac{1}{y} + 2 y \sqrt{d} \right) \\
&= \frac{1}{y^2} + 2 \sqrt{d}.
\end{aligned}
\]
由于 \(\frac{1}{y^2} \leq 1\)(因为 \(y \geq 1\)),则:
\[
\left| x^2 - d y^2 \right| \leq 1 + 2 \sqrt{d}.
\]

\textbf{推论2:} 对于每个无平方因子的整数 \(d > 1\),存在某个整数 \(k\) 满足 \(1 \leq |k| < 1 + 2\sqrt{d}\),且方程 \(x^2 - d y^2 = k\) 有无穷多整数解。  
证明:显然,\(x^2 - d y^2 = 0\) 的唯一整数解为 \((0, 0)\)。由推论 1,存在无穷多对 \((x, y)\)(\(y > 0\))满足 \(\left| x^2 - d y^2 \right| < 1 + 2\sqrt{d}\)。设 \(m = x^2 - d y^2\),则 \(m\) 是整数,且 \(|m| < 1 + 2\sqrt{d}\)。可能的 \(m\) 值是有限的:\(0, \pm 1, \pm 2, \dots, \pm [1 + 2\sqrt{d}]\)。由于 \((x, y)\) 有无穷多对,由抽屉原理,存在某个 \(k \neq 0\)(因为 \((0, 0)\) 仅对应 \(m = 0\))被无穷多次取到,即 \(x^2 - d y^2 = k\) 有无穷多解,且 \(1 \leq |k| < 1 + 2\sqrt{d}\)。

\textbf{证明(命题):} 由推论 2,存在某个 \(k\)(\(1 \leq |k| < 1 + 2\sqrt{d}\))使得 \(x^2 - d y^2 = k\) 有无穷多整数解。取其中两个正整数解 \((x_1, y_1)\) 和 \((x_2, y_2)\) 满足 \(x_1 \equiv x_2 \mod |k|\) 和 \(y_1 \equiv y_2 \mod |k|\)。则:
\[
(x_1 x_2 - d y_1 y_2)^2 - d (x_1 y_2 - x_2 y_1)^2 = (x_1^2 - d y_1^2)(x_2^2 - d y_2^2) = k^2.
\]
由于 \(x_1 \equiv x_2 \mod |k|\) 且 \(y_1 \equiv y_2 \mod |k|\),则 \(x_1 y_2 - x_2 y_1 \equiv 0 \mod |k|\),故 \(k\) 整除 \(x_1 y_2 - x_2 y_1\)。由上式,\(k\) 也整除 \((x_1 x_2 - d y_1 y_2) k\),从而整除 \(x_1 x_2 - d y_1 y_2\)。因此:
\[
\left( \frac{x_1 x_2 - d y_1 y_2}{k}, \frac{x_1 y_2 - x_2 y_1}{k} \right)
\]
是方程 \(x^2 - d y^2 = 1\) 的整数解。

令 \((x_0, y_0)\) 表示 \(x^2 - d y^2 = 1\) 的最小正整数解,即 \(x_0 + y_0 \sqrt{d} > 1\) 最小(注意到平凡解 \((1, 0)\) 对应 \(1 + 0 \cdot \sqrt{d} = 1\),我们取非平凡解)。我们声称,方程 \(x^2 - d y^2 = 1\) 的整数解为:
\[
\left\{ (x, y) \mid |x + y \sqrt{d}| = |x_0 + y_0 \sqrt{d}|^n, \text{对于某个 } n \in \mathbb{Z} \right\}.
\]
设 \((x, y)\) 是任意正整数解(由对称性,负解可类似处理)。若存在某个 \(n \geq 0\) 使得 \((x_0 + y_0 \sqrt{d})^n < x + y \sqrt{d} < (x_0 + y_0 \sqrt{d})^{n+1}\),则:
\[
1 < (x + y \sqrt{d}) (x_0 - y_0 \sqrt{d})^n < (x_0 + y_0 \sqrt{d}).
\]
令 \(x' + y' \sqrt{d} = (x + y \sqrt{d}) (x_0 - y_0 \sqrt{d})^n\)。由于 \((x_0 - y_0 \sqrt{d})^n = (x_0 + y_0 \sqrt{d})^{-n}\),且 \(x^2 - d y^2 = 1\),则:
\[
(x' + y' \sqrt{d})(x' - y' \sqrt{d}) = (x + y \sqrt{d})(x - y \sqrt{d}) (x_0 - y_0 \sqrt{d})^n (x_0 + y_0 \sqrt{d})^{-n} = 1,
\]
故 \(x' + y' \sqrt{d}\) 也满足 \(x'^2 - d y'^2 = 1\)。此外:
\[
1 < x' + y' \sqrt{d} < (x_0 + y_0 \sqrt{d}), \quad x' - y' \sqrt{d} = (x' + y' \sqrt{d})^{-1},
\]
则 \(0 < x' - y' \sqrt{d} < 1\)。于是:
\[
\begin{aligned}
x' &= \frac{1}{2} \left( (x' + y' \sqrt{d}) + (x' - y' \sqrt{d}) \right) > 0, \\
y' &= \frac{1}{2} \left( (x' + y' \sqrt{d}) - (x' - y' \sqrt{d}) \right) > 0,
\end{aligned}
\]
因此 \((x', y')\) 是一个正整数解,且 \(x' + y' \sqrt{d} < x_0 + y_0 \sqrt{d}\),这与 \((x_0, y_0)\) 的最小性矛盾。故 \(x + y \sqrt{d} = (x_0 + y_0 \sqrt{d})^n\)。

\textbf{结论:} 由命题,\(x^2 - d y^2 = 1\) 有非平凡解 \((x_0, y_0)\)。令 \(u_0 = x_0 + y_0 \sqrt{d}\),则 \(u_0^n = x_n + y_n \sqrt{d}\),且 \(x_n^2 - d y_n^2 = 1\)。由于 \(u_0 > 1\),\(n \in \mathbb{Z}\) 产生无穷多不同解,从而 \(\mathbb{Z}[\sqrt{d}]\) 的单位群由 \(\pm u_0^n\) 生成,包含无穷多个单位。


\item[7] \textbf{问题:} 证明对于每个无平方因子的整数 \(d > 1\),环 \(\mathbb{Z}[\sqrt{2}]\) 是欧几里得环。此外,证明其单位由 \(\pm (1 + \sqrt{2})^n, n \in \mathbb{Z}\) 给出,并确定其素元。

\textbf{解答:} 该问题包含三部分:证明 \(\mathbb{Z}[\sqrt{2}]\) 是欧几里得环,确定其单位群,并分类其素元。以下逐一解答。

\textbf{回忆:} 一个环 \(A\) 是欧几里得环,如果存在函数 \(\delta: A \to \mathbb{N}\),满足:(1) \(\delta(\alpha) = 0 \iff \alpha = 0\);(2) 对任意 \(\alpha, \beta \in A, \beta \neq 0\),存在 \(\kappa, \gamma \in A\) 使 \(\alpha = \kappa \beta + \gamma\) 且 \(\delta(\gamma) < \delta(\beta)\)。

\textbf{证明:} 为证明 \(\mathbb{Z}[\sqrt{2}]\) 是欧几里得环,定义范数 \(|N(a + b \sqrt{2})| = |a^2 - 2 b^2|\) 为欧几里得函数。考虑 \(\alpha = a + b \sqrt{2} \in \mathbb{Q}(\sqrt{2})\),选择 \(\gamma = x + y \sqrt{2} \in \mathbb{Z}[\sqrt{2}]\),使 \(|a - x| \leq \frac{1}{2}\),\(|b - y| \leq \frac{1}{2}\)。则:
\[
|N(\alpha - \gamma)| = |(a - x)^2 - 2 (b - y)^2| \leq |a - x|^2 + 2 |b - y|^2 \leq \frac{1}{4} + 2 \cdot \frac{1}{4} = \frac{3}{4} < 1.
\]
故 \(|N|\) 满足欧几里得条件,\(\mathbb{Z}[\sqrt{2}]\) 是欧几里得环。

验证 \(\pm (1 + \sqrt{2})^n\) 是单位:对于 \(u = 1 + \sqrt{2}\),\(N(u) = 1 - 2 = -1\)。其逆为 \(-1 + \sqrt{2}\),因:
\[
(1 + \sqrt{2})(-1 + \sqrt{2}) = -1 + 2 = 1.
\]
对 \(n \in \mathbb{Z}\),\(N((1 + \sqrt{2})^n) = (-1)^n = \pm 1\),故 \(\pm (1 + \sqrt{2})^n\) 均为单位。为证明其唯一性,设 \(u = a + b \sqrt{2}\) 为单位,则 \(N(u) = a^2 - 2 b^2 = \pm 1\)。取 \(u_0 = 1 + \sqrt{2}\) 为最小正单位(范数为 \(-1\),系数正),所有单位形如 \(\pm u_0^n\)。

\textbf{回忆:} 一个元素 \(p\) 称为素元,如果主理想 \((p)\) 是一个非零素理想。在唯一因子分解域中,素元恰好是不可约元素。
为了继续确定所有素元,我们需要一个引理。

\textbf{引理1:} 对于素数 \(p > 2\),二元一次方程:
\[
a^2 - 2 b^2 = p,
\]
有整数解当且仅当 \(p \equiv 1\) 或 \(7 \mod 8\)。  

\textbf{证明:} 对于任意整数 \(a, b\),\(a^2 - 2 b^2\) 不能模 8 为 3 或 5。模 8 检查:\(a^2 \equiv 0, 1, 4\); \(2 b^2 \equiv 0, 2\)。则 \(a^2 - 2 b^2 \equiv 0, 1, 2, 4, 6, 7\)。故不可能为 3 或 5。为了证明“当且”部分,我们只需证明这样的 \(p\) 在 \(\mathbb{Z}[\sqrt{2}]\) 中不是素元。设 \(p = \alpha \beta\),则 \(N(\alpha) N(\beta) = N(p) = p^2\),因此 \(N(\alpha) = \pm p\) 或 \(\pm p^2\),其中 \(\alpha = a + b \sqrt{2}\),于是 \(p = N(\alpha) = a^2 - 2 b^2\)。

为此,我们验证当 \(p \equiv 1\) 或 \(7 \mod 8\) 时,同余式 \(x^2 \equiv 2 \mod p\) 有解。若如此,则 \(p \mid x^2 - 2 = (x + \sqrt{2})(x - \sqrt{2})\)。但 \(\frac{x + \sqrt{2}}{p}\) 和 \(\frac{x - \sqrt{2}}{p}\) 都不在 \(\mathbb{Z}[\sqrt{2}]\) 中,因此 \(p\) 不能是素元。

为了证明 2 是模 \(p\) 的二次剩余,即同余式 \(x^2 \equiv 2 \mod p\) 有解,当 \(p \equiv 1\) 或 \(7 \mod 8\) 时,我们引用 Legendre 符号和 Gauss 引理。

\textbf{Legendre 符号:} 一个整数 \(a\) 称为模 \(n\) 的二次剩余,如果合同式 \(x^2 \equiv a \mod n\) 有解。我们定义模 \(n\) 的 Legendre 符号如下:
\[
\left(\frac{a}{n}\right) := 
\begin{cases} 
1 & \text{若 } a \text{ 是模 } n \text{ 的二次剩余且 } a \not\equiv 0 \mod p, \\
0 & \text{若 } a \equiv 0 \mod p, \\
-1 & \text{若 } a \text{ 是模 } n \text{ 的二次非剩余}.
\end{cases}
\]

\textbf{引理2:} 设 \(p\) 是一个奇素数,\(a\) 是一个整数,则:
\[
\left(\frac{a}{p}\right) \equiv a^{\frac{p-1}{2}} \mod p.
\]
\textbf{证明:} 我们将正余数集 \(1, 2, \cdots, p-1\) 中的整数配对如下:如果 \(x y \equiv a \mod p\),则将 \(x\) 与 \(y\) 配对。

若 \(a\) 是二次剩余,则存在某个 \(x_0\) 在正余数集中,使得 \(x_0^2 \equiv a \mod p\)。在这种情况下,同余式 \(x^2 \equiv a \mod p\) 在正余数集中恰有两解:\(x_0\) 和 \(p - x_0\)。因此配对提供 \(\frac{p-3}{2}\) 对和两个单元素,它们的乘积为:
\[
(p-1)! \equiv a^{\frac{p-3}{2}} x_0 (p - x_0) \equiv -a^{\frac{p-1}{2}} \mod p.
\]
由于 \((p-1)! \equiv -1 \mod p\)(Wilson 定理),得 \(a^{\frac{p-1}{2}} \equiv 1 \mod p\)。
若 \(a\) 是二次非剩余,则配对提供 \(\frac{p-1}{2}\) 对,它们的乘积为:
\[
(p-1)! \equiv a^{\frac{p-1}{2}} \mod p.
\]
于是 \(a^{\frac{p-1}{2}} \equiv -1 \mod p\)。结论为 \(\left(\frac{a}{p}\right) \equiv a^{\frac{p-1}{2}} \mod p\)。

\textbf{引理 (Gauss 引理)} 设 \(p\) 是一个奇素数,\(a\) 与 \(p\) 互质,则:
\[
\left(\frac{a}{p}\right) \equiv (-1)^\mu \mod p,
\]
其中 \(\mu\) 是 \(a, 2a, \cdots, \frac{p-1}{2} a\) 模 \(p\) 的绝对余数中负整数的个数。  

\textbf{证明:} 设 \(r_1, r_2, \cdots, r_\tau\) 是 \(a, 2a, \cdots, \frac{p-1}{2} a\) 模 \(p\) 的正绝对余数,而 \(s_1, s_2, \cdots, s_\mu\) 是负的。则 \(\tau + \mu = \frac{p-1}{2}\)。注意 \(r_1, r_2, \cdots, r_\tau, -s_1, -s_2, \cdots, -s_\mu\) 是互异的,因此它们是 \(1, 2, \cdots, \frac{p-1}{2}\) 的一种排列。因此有:
\[
r_1 r_2 \cdots r_\tau (-s_1)(-s_2) \cdots (-s_\mu) \equiv \left(\frac{p-1}{2}\right)! \mod p.
\]
但:
\[
r_1 r_2 \cdots r_\tau s_1 s_2 \cdots s_\mu \equiv \left(\frac{p-1}{2}\right)! a^{\frac{p-1}{2}} \mod p.
\]
于是 \(a^{\frac{p-1}{2}} \equiv (-1)^\mu \mod p\). 结果由引理2得出。

\textbf{备注:} 使用这个 Gauss 引理,可以确定 \(\left(\frac{a}{p}\right)\)。确实,\(2, 4, \cdots, p-1\) 模 \(p\) 的绝对余数是 \(2, 4, \cdots, 2 \left[\frac{p-1}{4}\right]\) 和 \(2 \left[\frac{p-1}{4}\right] - p, \cdots, -1\)。于是 \(\mu = \frac{p-1}{2} - \left[\frac{p-1}{4}\right]\),我们得出 \(\left(\frac{2}{p}\right) = (-1)^{\frac{p-1}{2} - \left[\frac{p-1}{4}\right]} = (-1)^{\frac{p^2 - 1}{8}}\)。

\textbf{命题1:} \(\mathbb{Z}[\sqrt{2}]\) 的素元 \(\pi\),在等价类元素(associated elements)下,可由如下给出:
\begin{enumerate}
    \item \(\pi = \sqrt{2}\),
    \item \(\pi = a + b \sqrt{2}\) 其中 \(a^2 - 2 b^2 = p, p \equiv 1, 7 \mod 8, a > b \sqrt{2} > 0\),
    \item \(\pi = p, p \equiv 3, 5 \mod 8\).
\end{enumerate}
\textbf{证明:} 设 \(\pi\) 是一个素元。我们将 \(N(\pi)\) 分解为素数,则 \(\pi\) 必须除以其中之一,设为 \(p\)。则 \(N(\pi) \mid N(p) = p^2\),因此 \(N(\pi) = \pm p\) 或 \(\pm p^2\)。为了确定 \(\pi\),需确定 \(a^2 - 2 b^2 = p\):若无正整数解,则 \(\pi = p\) 是素元,这是情况 3;若有正整数解 \((a, b)\) 则 \(\pi = a + b \sqrt{2}\) 是素元,这是情况 1 和 2。结果由\textbf{引理1}得出。

\textbf{总结:} 如上证明了\(\mathbb{Z}[\sqrt{2}]\) 是欧几里得环,单位为 \(\pm (1 + \sqrt{2})^n\),素元如\textbf{命题1}。

\end{enumerate}

\section{第二章:整性}

\subsection{练习题}

\begin{enumerate}

\item[1] 
\textbf{问题:} 判断 \(\frac{3 + 2 \sqrt{6}}{1 - \sqrt{6}}\) 是否为代数整数。\\
\textbf{解答:} 设 \(\theta = \frac{3 + 2 \sqrt{6}}{1 - \sqrt{6}}\)。计算得(分母有理化):
\[
\theta = \frac{3 + 2 \sqrt{6}}{1 - \sqrt{6}} \cdot \frac{1 + \sqrt{6}}{1 + \sqrt{6}} = \frac{(3 + 2 \sqrt{6})(1 + \sqrt{6})}{1 - 6} = \frac{3 + 3 \sqrt{6} + 2 \sqrt{6} + 12}{-5} = -\frac{15 + 5 \sqrt{6}}{5} = -(3 + \sqrt{6})
\]
验证 \(\theta\) 是否满足整系数单项式方程。计算:
\[
\theta^2 + 6\theta + 3 = (3 + \sqrt{6})^2 - 6(3 + \sqrt{6}) + 3 = 9 + 6 \sqrt{6} + 6 - 18 - 6 \sqrt{6} + 3 = 0
\]
因此,\(\theta\) 满足整系数方程 \(x^2 + 6x + 3 = 0\),故 \(\theta\) 是代数整数。


% \item[2] \textbf{问题:} 证明若整环 \(A\) 是整闭(integrally closed)的,则其多项式环 \(A[t]\) 也是整闭的。

% \textbf{解答:} 设 \(K\) 为 \(A\) 的分式域。需证明 \(A[t]\) 在其分式域 \(K(t)\) 中整闭,即若 \(f(t) \in K(t)\) 对 \(A[t]\) 整,则 \(f(t) \in A[t]\)。已知 \(K[t]\) 是主理想域(PID),由命题 2.1 或 2.2,\(K[t]\) 整闭。注意到 \(K(t)\) 是 \(A[t]\) 和 \(K[t]\) 的共同分式域。

% 设 \(f(t) \in K(t)\) 对 \(A[t]\) 整,则对 \(K[t]\) 也整,故 \(f(t) \in K[t]\)。存在 \(a_{n-1}(t), \dots, a_0(t) \in A[t]\),使得:
% \[
% f(t)^n + a_{n-1}(t) f(t)^{n-1} + \cdots + a_0(t) = 0.
% \]
% 取整数 \(m\) 大于 \(a_{n-1}(t), \dots, a_0(t)\) 和 \(f(t)\) 的次数。令 \(h(t) = t^m - f(t)\),则 \(h(t)\) 对 \(A[t]\) 整,且为首一多项式。代入得:
% \[
% h(t)^n + a_{n-1}(t) h(t)^{n-1} + \cdots + a_0(t) = 0.
% \]
% 设 \(g(t) = h(t)^{n-1} + a_{n-1}(t) h(t)^{n-2} + \cdots + a_1(t)\),则:
% \[
% h(t) g(t) = -a_0(t).
% \]
% 因 \(h(t)\) 和 \(-a_0(t)\) 均为首一,且 \(-a_0(t) \in A[t]\),由引理 2.4(Gauss 引理的推广),\(h(t)\) 的系数对 \(A\) 整。因 \(A\) 整闭,\(h(t) \in A[t]\)。故 \(f(t) = t^m - h(t) \in A[t]\)。\(A[t]\) 整闭得证。

% \item[2.1] \textbf{问题:} 证明每个唯一因子分解整环(UFD)是整闭的。

% \textbf{解答:} 设 \(A\) 为 UFD,\(K\) 为其分式域。任取 \(a/b \in K\)(\(a, b \in A\),\(b \neq 0\))对 \(A\) 整,需证 \(a/b \in A\)。因 \(a/b\) 对 \(A\) 整,存在 \(a_{n-1}, \dots, a_0 \in A\),使得:
% \[
% (a/b)^n + a_{n-1} (a/b)^{n-1} + \cdots + a_0 = 0.
% \]
% 等式两边乘以 \(b^n\):
% \[
% a^n + a_{n-1} a^{n-1} b + \cdots + a_0 b^n = 0 \implies a^n = - (a_{n-1} a^{n-1} b + \cdots + a_0 b^n).
% \]
% 右边被 \(b\) 整除,故 \(b \mid a^n\)。在 UFD 中,设 \(b = p_1^{e_1} \cdots p_k^{e_k}\),若某质元 \(p_i \mid b\),则 \(p_i^{e_i n} \mid a^n\)。若 \(p_i \mid a\),则 \(a/b\) 非分式,矛盾。故 \(p_i \nmid a\)。由唯一因子分解,\(b\) 的每个质因子次数在 \(a^n\) 中为 0,即 \(b\) 为单位。故 \(a/b \in A\)。\(A\) 整闭得证。

% \item[2.2] \textbf{问题:} 证明每个 Dedekind 域是整闭的。

% \textbf{解答:} 设 \(\mathcal{O}\) 为 Dedekind 整环,\(K\) 为其分式域。任取 \(r \in K\) 对 \(\mathcal{O}\) 整,需证 \(r \in \mathcal{O}\)。因 \(r\) 对 \(\mathcal{O}\) 整,存在 \(n\) 使得 \(r^n \in (r^{n-1}, r^{n-2}, \dots, 1)\)。记理想 \(I = (r, 1)\),则:
% \[
% I^n = (r^n, r^{n-1}, \dots, 1) = (r^{n-1}, r^{n-2}, \dots, 1) = I^{n-1}.
% \]
% 因 \(\mathcal{O}\) 为 Dedekind 整环,\(I\) 可逆,乘以 \(I^{-1}\) 得 \(I = \mathcal{O}\)。故 \(r \in I = \mathcal{O}\)。\(\mathcal{O}\) 整闭得证。

% \item[2.3] \textbf{问题(Gauss 引理):} 设 \(A\) 为 UFD,\(K\) 为其分式域。若 \(f, g \in K[t]\) 为首一多项式,\(g \mid f\),且 \(f \in A[t]\),证明 \(g \in A[t]\)。

% \textbf{解答:} 设 \(f = g h\),\(h \in K[t]\) 首一。因 \(A\) 为 UFD,存在 \(a, b \in A\) 使 \(a g, b h \in A[t]\),且 \(a g\) 和 \(b h\) 的系数无公共质因子(即内容为 1)。则 \(a b f = (a g)(b h)\)。因 \(f \in A[t]\),由 Gauss 引理(标准形式),\(a g \in A[t]\) 的系数无非单位公共因子,因 \(g\) 首一,\(a\) 为单位,故 \(g \in A[t]\)。得证。

% \item[2.4] \textbf{问题:} 设 \(A\) 为环,\(B\) 为 \(A\)-代数,\(f, g \in B[t]\) 为首一多项式,\(g \mid f\)。若 \(f\) 的系数对 \(A\) 整,证明 \(g\) 的系数也对 \(A\) 整。

% \textbf{解答:} 设 \(A' \subset B\) 为 \(f\) 系数生成的 \(A\)-子代数,则 \(A'\) 对 \(A\) 整。考虑 \(f\) 在分裂域中的根,它们对 \(A'\) 整,因整性传递,对 \(A\) 整。\(g\) 的根为 \(f\) 的根子集,故对 \(A\) 整,其系数(对称多项式)也对 \(A\) 整。得证。

% \item[2.5] \textbf{问题:} 设 \(A\) 为整域,\(A'\) 为其整闭包,证明 \(A[t]\) 的整闭包为 \(A'[t]\)。

% \textbf{解答:} 需证 \(A'[t]\) 是 \(A[t]\) 在 \(K(t)\) 中的整闭包。

% (1) \(A'[t] \subset K(t)\) 对 \(A[t]\) 整:任取 \(f(t) = \sum a_i t^i \in A'[t]\),\(a_i \in A'\) 对 \(A\) 整,存在方程 \(a_i^n + b_{n-1} a_i^{n-1} + \cdots + b_0 = 0\),\(b_j \in A\)。则 \(f(t)\) 满足:
% \[
% f(t)^n + b_{n-1} f(t)^{n-1} + \cdots + b_0 = 0,
% \]
% 系数在 \(A[t]\) 中,故 \(f(t)\) 对 \(A[t]\) 整。

% (2) 若 \(f(t) \in K(t)\) 对 \(A[t]\) 整,则 \(f(t) \in A'[t]\):由问题 2,\(f(t) \in A[t]\)。设 \(f(t) = \sum c_i t^i\),\(c_i \in K\) 对 \(A\) 整,因 \(A'\) 为 \(A\) 在 \(K\) 中的整闭包,\(c_i \in A'\)。故 \(f(t) \in A'[t]\)。得证。


\item[2] \textbf{问题:} 证明若整环 \(A\) 是整闭(integrally closed)的,则其多项式环 \(A[t]\) 也是整闭的。此外,证明每个唯一因子分解整环(UFD)和每个 Dedekind 域是整闭的,并进一步说明 \(A[t]\) 的整闭包与 \(A\) 的整闭包之间的关系。

\textbf{解答:} 该问题包含多个部分:证明 \(A[t]\) 整闭,验证 UFD 和 Dedekind 域的整闭性,及 \(A[t]\) 整闭包的性质。以下逐一解答。

\textbf{回忆:} 一个整环 \(A\) 是整闭的,若其分式域 \(K\) 中对 \(A\) 整的元素均在 \(A\) 中。即,若 \(x \in K\) 满足 \(x^n + a_{n-1} x^{n-1} + \cdots + a_0 = 0\)(\(a_i \in A\)),则 \(x \in A\)。

\textbf{证明:} 设 \(K\) 为 \(A\) 的分式域,需证明 \(A[t]\) 在其分式域 \(K(t)\) 中整闭,即若 \(f(t) \in K(t)\) 对 \(A[t]\) 整,则 \(f(t) \in A[t]\)。已知 \(K[t]\) 是主理想域(PID),由后述\textbf{命题 1 或 2},\(K[t]\) 整闭,且 \(K(t)\) 是 \(A[t]\) 和 \(K[t]\) 的共同分式域。

设 \(f(t) \in K(t)\) 对 \(A[t]\) 整,则对 \(K[t]\) 也整,故 \(f(t) \in K[t]\)。存在 \(a_{n-1}(t), \dots, a_0(t) \in A[t]\),使得:
\[
f(t)^n + a_{n-1}(t) f(t)^{n-1} + \cdots + a_0(t) = 0.
\]
取整数 \(m\) 大于 \(a_{n-1}(t), \dots, a_0(t)\) 和 \(f(t)\) 的次数。令 \(h(t) = t^m - f(t)\),则 \(h(t)\) 对 \(A[t]\) 整,且为首一多项式。代入得:
\[
h(t)^n + a_{n-1}(t) h(t)^{n-1} + \cdots + a_0(t) = 0.
\]
设 \(g(t) = h(t)^{n-1} + a_{n-1}(t) h(t)^{n-2} + \cdots + a_1(t)\),则:
\[
h(t) g(t) = -a_0(t).
\]
因 \(h(t)\) 和 \(-a_0(t)\) 均为首一,且 \(-a_0(t) \in A[t]\),由\textbf{引理2},\(h(t)\) 的系数对 \(A\) 整。因 \(A\) 整闭,\(h(t) \in A[t]\)。故 \(f(t) = t^m - h(t) \in A[t]\)。\(A[t]\) 整闭得证。

\textbf{命题1:} 每个唯一因子分解整环(UFD)是整闭的。

\textbf{证明:} 设 \(A\) 为 UFD,\(K\) 为其分式域。任取 \(a/b \in K\)(\(a, b \in A\),\(b \neq 0\))对 \(A\) 整,存在 \(a_{n-1}, \dots, a_0 \in A\),使得:
\[
(a/b)^n + a_{n-1} (a/b)^{n-1} + \cdots + a_0 = 0.
\]
乘以 \(b^n\):
\[
a^n = - (a_{n-1} a^{n-1} b + \cdots + a_0 b^n).
\]
右边被 \(b\) 整除,故 \(b \mid a^n\)。设 \(b = p_1^{e_1} \cdots p_k^{e_k}\),若某质元 \(p_i \mid b\),则 \(p_i^{e_i n} \mid a^n\)。若 \(p_i \mid a\),则 \(a/b\) 非分式,矛盾。故 \(p_i \nmid a\)。由唯一因子分解,\(b\) 为单位,\(a/b \in A\)。\(A\) 整闭得证。

\textbf{命题2:} 每个 Dedekind 域是整闭的。

\textbf{证明:} 设 \(\mathcal{O}\) 为 Dedekind 域,\(K\) 为其分式域。任取 \(r \in K\) 对 \(\mathcal{O}\) 整,存在 \(n\) 使得 \(r^n \in (r^{n-1}, r^{n-2}, \dots, 1)\)。记理想 \(I = (r, 1)\),则:
\[
I^n = (r^n, r^{n-1}, \dots, 1) = (r^{n-1}, r^{n-2}, \dots, 1) = I^{n-1}.
\]
因 \(\mathcal{O}\) 为 Dedekind 域,\(I\) 可逆,乘以 \(I^{-1}\) 得 \(I = \mathcal{O}\)。故 \(r \in \mathcal{O}\)。\(\mathcal{O}\) 整闭得证。

\textbf{引理1(Gauss 引理):} 设 \(A\) 为 UFD,\(K\) 为其分式域。若 \(f, g \in K[t]\) 为首一多项式,\(g \mid f\),且 \(f \in A[t]\),则 \(g \in A[t]\)。

\textbf{证明:} \textbf{TODO 待确认} 设 \(f = g h\),\(h \in K[t]\) 首一。存在 \(a, b \in A\) 使 \(a g, b h \in A[t]\),且 \(a g\) 和 \(b h\) 的系数无公共质因子)。则 \(a b f = (a g)(b h)\)。因 \(f \in A[t]\),由 Gauss 引理标准形式,\(a g \in A[t]\) 的系数无非单位公共因子,因 \(g\) 首一,\(a\) 为单位,\(g \in A[t]\)。得证。

\textbf{引理2:} 设 \(A\) 为环,\(B\) 为 \(A\)-代数,\(f, g \in B[t]\) 为首一多项式,\(g \mid f\)。若 \(f\) 的系数对 \(A\) 整,则 \(g\) 的系数也对 \(A\) 整。

\textbf{证明:} 设 \(A' \subset B\) 为 \(f\) 系数生成的 \(A\)-子代数,则 \(A'\) 对 \(A\) 整。\(f\) 在分裂域中的根对 \(A'\) 整,因整性传递,对 \(A\) 整。\(g\) 的根为 \(f\) 的根子集,故对 \(A\) 整,其系数也对 \(A\) 整。得证。

\textbf{命题3:} 设 \(A\) 为整域,\(A'\) 为其在 \(K\) 中的整闭包,则 \(A[t]\) 在 \(K(t)\) 中的整闭包为 \(A'[t]\)。

\textbf{证明:} (1) \(A'[t]\) 对 \(A[t]\) 整:任取 \(f(t) = \sum a_i t^i \in A'[t]\),\(a_i \in A'\) 对 \(A\) 整,满足 \(a_i^n + b_{n-1} a_i^{n-1} + \cdots + b_0 = 0\)(\(b_j \in A\))。则:
\[
f(t)^n + b_{n-1} f(t)^{n-1} + \cdots + b_0 = 0,
\]
系数在 \(A[t]\) 中,\(f(t)\) 对 \(A[t]\) 整。

(2) 若 \(f(t) \in K(t)\) 对 \(A[t]\) 整,则 \(f(t) \in A'[t]\):由前述,\(f(t) \in A[t]\)。设 \(f(t) = \sum c_i t^i\),\(c_i \in K\) 对 \(A\) 整,因 \(A'\) 为整闭包,\(c_i \in A'\)。故 \(f(t) \in A'[t]\)。得证。

\textbf{总结:} 如上证明了若 \(A\) 整闭,则 \(A[t]\) 整闭;UFD 和 Dedekind 域均整闭(\textbf{命题 1 和 2});\(A[t]\) 的整闭包为 \(A'[t]\)(\textbf{命题3})。\textbf{引理1 和 2} 提供了关键工具。


% \item[3] 
% \textbf{问题:} 在多项式环 \(A = \mathbb{Q}[X, Y]\) 中,考虑主理想 \(p = (X^2 - Y^3)\)。证明 \(p\) 是素理想,但 \(A/p\) 不是整闭的。\\
% \textbf{解答:} 为证 \(p\) 是素理想,需证明 \(X^2 - Y^3\) 在 \(\mathbb{Q}\) 上不可约,显然成立。  
% 为证 \(A/p\) 非整闭,考虑分式域中方程 \(T^2 - Y = 0\),其解 \(T = \frac{X}{Y}\) 在 \(A/p\) 的分式域中,但不在 \(A/p\) 中。因在 \(A/p\) 中,\(X^2 = Y^3\),故 \(T^2 = \left(\frac{X}{Y}\right)^2 = \frac{X^2}{Y^2} = \frac{Y^3}{Y^2} = Y\),但 \(\frac{X}{Y} \notin A/p\)。

\item[4] 
\textbf{问题:} 设 \(D\) 为不等于 0 或 1 的无平方因子有理整数,\(K = \mathbb{Q}(\sqrt{D})\) 为二次数域,\(d\) 为其判别式。证明:
\[
\begin{cases}
d = D, & \text{若 } D \equiv 1 \pmod{4}, \\
d = 4D, & \text{若 } D \equiv 2 \text{ 或 } 3 \pmod{4},
\end{cases}
\]
且 \(K\) 的整基在第二种情况下为 \(\{1, \sqrt{D}\}\),第一种情况下为 \(\left\{1, \frac{1}{2}(1 + \sqrt{D})\right\}\),且在两种情况下均为 \(\left\{1, \frac{1}{2}(d + \sqrt{d})\right\}\)。\\
\textbf{解答:} 设 \(a + b \sqrt{D} \in \mathbb{Q}(\sqrt{D})\) 为代数整数,其最小多项式为 \(x^2 - 2a x + (a^2 - D b^2)\)。故 \(2a \in \mathbb{Z}\),\(a^2 - D b^2 \in \mathbb{Z}\)。  
若 \(a \in \mathbb{Z}\),则 \(D b^2 \in \mathbb{Z}\),因 \(D\) 无平方因子,\(b \in \mathbb{Z}\)。  
若 \(a \notin \mathbb{Z}\),则 \(2a\) 为奇数,\(D (2b)^2 \equiv 1 \pmod{4}\),因 \(D\) 无平方因子,\(2b \in \mathbb{Z}\),且 \(D \equiv 1 \pmod{4}\)。  
因此,\(\mathcal{O}_K = \mathbb{Z} + \frac{1}{2}(1 + \sqrt{D}) \mathbb{Z}\)(若 \(D \equiv 1 \pmod{4}\)),\(\mathcal{O}_K = \mathbb{Z} + \sqrt{D} \mathbb{Z}\)(若 \(D \equiv 2 \text{ 或 } 3 \pmod{4}\))。  

计算判别式 \(d\)。所有嵌入 \(K \to \mathbb{C}\) 为恒等映射及 \(\sigma: a + b \sqrt{D} \mapsto a - b \sqrt{D}\)。  
若 \(D \equiv 1 \pmod{4}\),则:
\[
d = d\left(1, \frac{1}{2}(1 + \sqrt{D})\right) = \det\begin{pmatrix} 1 & \frac{1}{2}(1 + \sqrt{D}) \\ 1 & \frac{1}{2}(1 - \sqrt{D}) \end{pmatrix}^2 = D
\]
若 \(D \equiv 2 \text{ 或 } 3 \pmod{4}\),则:
\[
d = d(1, \sqrt{D}) = \det\begin{pmatrix} 1 & \sqrt{D} \\ 1 & -\sqrt{D} \end{pmatrix}^2 = 4D
\]
两情况下,\(\mathcal{O}_K = \mathbb{Z} + \frac{1}{2}(d + \sqrt{d}) \mathbb{Z}\)。  


\item[4.1] \textbf{问题:} 设 \( A \subset B \) 是环的扩展,且 \( B \) 作为 \( A \)-模是秩为 \( m \) 的自由模。给定 \( B \) 中的元素 \(\beta_1, \cdots, \beta_m\),定义此基的判别式,并说明当基变换时判别式的变化规律。进一步说明判别式 \( d(B / A) \) 如何作为 \( A / A^{*2} \) 中的元素,以及在更一般的情况下如何定义 \( d(L / K) \)。

\textbf{解答:} 设 \( A \subset B \) 是环的扩展,且 \( B \) 作为 \( A \)-模是秩为 \( m \) 的自由模。对于 \( B \) 中的元素 \(\beta_1, \cdots, \beta_m\),其判别式定义为:
\[
d(\beta_1, \cdots, \beta_m) = \det\left(\operatorname{Tr}_{B \mid A}(\beta_i \beta_j)\right),
\]
其中 \(\operatorname{Tr}_{B \mid A}\) 是 \( B \) 相对于 \( A \) 的迹映射。可以验证,\((\alpha, \beta) \mapsto \operatorname{Tr}_{B \mid A}(\alpha \beta)\) 是一个对称双线性形式。因此,若 \(\gamma_j = \sum_i a_{ji} \beta_i\)(其中 \( a_{ij} \in A \)),则:
\[
d(\gamma_1, \cdots, \gamma_m) = \det(a_{ij})^2 d(\beta_1, \cdots, \beta_m).
\]
若 \(\beta_1, \cdots, \beta_m\) 和 \(\gamma_1, \cdots, \gamma_m\) 均为 \( B \) 的基,则 \(\det(a_{ij})\) 是 \( A \) 中的单位,故判别式在单位平方乘积的意义下不变。因此,判别式可视为 \( A / A^{*2} \) 中的元素,记为 \( d(B / A) \),称为 \( B \) 相对于 \( A \) 的判别式。

更一般地,设 \( K \) 是 \( A \) 的分式域,\( L \) 是 \( K \) 的度为 \( m \) 的扩展。若 \( A \) 在 \( L \) 中的整闭包 \( B \) 是 \( A \) 上秩为 \( m \) 的自由模,则 \( d(B / A) \) 表示 \( d(L / K) \)。此外,若 \( L \mid K \) 是可分的,则 \( d(L / K) \neq 0 \)。

\item[4.1.1] \textbf{问题:} 在问题4.1中, 当 \( A = \mathbb{Z} \) 时,说明判别式 \( d(B / A) \) 是一个明确定义的整数,并解释为什么可以省略基环 \(\mathbb{Z}\),直接记为 \( d(B) \)。对于数域 \( K \) 是 \(\mathbb{Q}\) 上度为 \( m \) 的扩展的情况,说明环 \(\mathcal{O}_K\) 在 \(\mathbb{Z}\) 上是秩 \( m \) 的自由模,并定义 \( d_K \) 作为 \( K \) 的判别式。

\textbf{解答:} 当 \( A = \mathbb{Z} \) 时,\(\mathbb{Z}\) 中单位只有 \( \pm 1 \),其平方仅为 1。因此,判别式 \( d(B / A) \) 作为一个在 \(\mathbb{Z} / \mathbb{Z}^{*2} \) 中的元素,只可能是整数本身,不受单位平方的模除影响,故 \( d(B / A) \) 是明确定义的整数。在这种情况下,我们可以省略基环 \(\mathbb{Z}\),直接记为 \( d(B) \)。

对于数域 \( K \) 是 \(\mathbb{Q}\) 上度为 \( m \) 的扩展,\(\mathcal{O}_K\) 是 \( K \) 中整环,它在 \(\mathbb{Z}\) 上是秩 \( m \) 的自由模。因此,\( d(\mathcal{O}_K) \) 是一个明确定义的整数。我们定义 \( d(K / \mathbb{Q}) \) 作为 \(\mathbb{Q}^*/\mathbb{Q}^{*2}\) 中的元素,由 \( d(\mathcal{O}_K) \) 表示,并将此整数记为 \( d_K \),称为数域 \( K \) 的判别式。

\item[4.2] \textbf{问题:} 设 \(\mathfrak{a} \subset \mathfrak{a}' \) 是数域 \( K \) 中两个非零的有限生成 \(\mathcal{O}_K\)-子模,证明指数(index) \( (\mathfrak{a}' : \mathfrak{a}) \) 是有限的,并且满足关系:
\[
d(\mathfrak{a}) = (\mathfrak{a}' : \mathfrak{a})^2 d(\mathfrak{a}').
\]

\textbf{解答:} 设 \(\beta_1, \cdots, \beta_m\) 是 \(\mathfrak{a}'\) 的一个整基。因为 \(\mathfrak{a} \subset \mathfrak{a}'\) 是 \(\mathbb{Z}\)-模,根据有限生成 \(\mathbb{Z}\)-模的基本定理,存在整数 \( a_1, \cdots, a_m \) 满足 \( a_i \mid a_{i+1} \)(\( i = 1, \cdots, m-1 \)),使得 \( a_1 \beta_1, \cdots, a_m \beta_m \) 是 \(\mathfrak{a}\) 的整基。此外,\(\mathfrak{a}' / \mathfrak{a} \cong \mathbb{Z}/(a_1) \oplus \cdots \oplus \mathbb{Z}/(a_m)\),因此索引 \( (\mathfrak{a}' : \mathfrak{a}) = a_1 a_2 \cdots a_m \) 是有限的。

现在计算判别式:
\[
d(\mathfrak{a}) = d(a_1 \beta_1, \cdots, a_m \beta_m) = \det(T)^2 d(\beta_1, \cdots, \beta_m) = \det(T)^2 d(\mathfrak{a}'),
\]
其中基变换矩阵 \( T = \operatorname{diag}(a_1, a_2, \cdots, a_m) \),因此 \(\det(T) = a_1 a_2 \cdots a_m = (\mathfrak{a}' : \mathfrak{a})\)。于是:
\[
d(\mathfrak{a}) = (\mathfrak{a}' : \mathfrak{a})^2 d(\mathfrak{a}').
\]
证毕。

% \item[4.3] \textbf{问题:} 利用 Exercise 7 的结论,说明数域的判别式 \( d \equiv 0 \text{ 或 } 1 \pmod{4} \),并推导出当 \( D \equiv 2 \text{ 或 } 3 \pmod{4} \) 时,\( d \) 必须等于 \( 4D \)。

% \textbf{解答:} 由 Exercise 7(Stickelberger 判别式关系)可知,代数数域 \( K \) 的判别式 \( d_K \) 总是满足 \( d_K \equiv 0 \text{ 或 } 1 \pmod{4} \)。在 Exercise 4 中,对于二次数域 \( K = \mathbb{Q}(\sqrt{D}) \):
% - 当 \( D \equiv 2 \text{ 或 } 3 \pmod{4} \) 时,判别式 \( d = 4D \)(由 \( d(1, \sqrt{D}) = 4D \) 计算得出)。
% - 检查模 4 的余数:
%   - 若 \( D \equiv 2 \pmod{4} \),则 \( 4D \equiv 8 \equiv 0 \pmod{4} \);
%   - 若 \( D \equiv 3 \pmod{4} \),则 \( 4D \equiv 12 \equiv 0 \pmod{4} \)。

% 在两种情况下,\( 4D \equiv 0 \pmod{4} \),符合 \( d_K \equiv 0 \text{ 或 } 1 \pmod{4} \) 的要求。此外,由 Exercise 4 的证明,\(\mathcal{O}_K = \mathbb{Z}[\sqrt{D}]\) 时 \( d = 4D \),且没有其他可能性(因为若索引不为 1,会导致矛盾)。因此,当 \( D \equiv 2 \text{ 或 } 3 \pmod{4} \) 时,\( d \) 必须等于 \( 4D \)。


\item[5] 
\textbf{问题:} 证明 \(\left\{1, \sqrt[3]{2}, \sqrt[3]{2}^2\right\}\) 是 \(\mathbb{Q}(\sqrt[3]{2})\) 的整基。\\
\textbf{解答1:} 首先计算基 \(\left\{1, \sqrt[3]{2}, \sqrt[3]{2}^2\right\}\) 的判别式。\(\mathbb{Q}(\sqrt[3]{2})\) 到 \(\mathbb{C}\) 的嵌入为 \(\sigma_1 = \text{id}\)、\(\sigma_2: \sqrt[3]{2} \mapsto \sqrt[3]{2} \omega\)、\(\sigma_3: \sqrt[3]{2} \mapsto \sqrt[3]{2} \omega^2\),其中 \(\omega\) 为三次单位根。则:
\[
\begin{aligned}
d\left(1, \sqrt[3]{2}, \sqrt[3]{2}^2\right) &= \det\begin{pmatrix}
1 & \sqrt[3]{2} & \sqrt[3]{2}^2 \\
1 & \sqrt[3]{2} \omega & \sqrt[3]{2}^2 \omega^2 \\
1 & \sqrt[3]{2} \omega^2 & \sqrt[3]{2} \omega
\end{pmatrix}^2 \\
&= \left( (\sqrt[3]{2} - \sqrt[3]{2} \omega)(\sqrt[3]{2} \omega - \sqrt[3]{2} \omega^2)(\sqrt[3]{2} - \sqrt[3]{2} \omega^2) \right)^2 \\
&= 4 (1 - \omega)^2 (\omega - \omega^2)^2 (1 - \omega^2)^2 \\
&= 4 (1 - \omega)^6 = 4 (-3 \omega)^3 = -108
\end{aligned}
\]
\textbf{命题1:} 设 \(\mathfrak{a} \subset \mathfrak{a}' \) 是数域 \( K \) 中两个非零的有限生成 \(\mathcal{O}_K\)-子模,则指数(index) \( (\mathfrak{a}' : \mathfrak{a}) \) 是有限的,并且满足关系:
\[
d(\mathfrak{a}) = (\mathfrak{a}' : \mathfrak{a})^2 d(\mathfrak{a}').
\]

由\textbf{命题1}可得:
\(\left(\mathcal{O}_K : \mathbb{Z}[\sqrt[3]{2}]\right)^2 d_K = -108 = -2^2 \cdot 3^3\)。设指数为 \(m\),则 \(m = 1, 2, 3\) 或 6。

\textbf{Stickelberger 判别式关系:}代数域 \(K\) 的判别式 \(d_K\) 总是 \(\equiv 0 \text{ 或 } 1 \pmod{4}\)。

由\textbf{Stickelberger 判别式关系}可得:
\(m = 1\) 或 3。若 \(m = 3\),则 \(3 \mathcal{O}_K \subset \mathbb{Z}[\sqrt[3]{2}]\),存在 \(\alpha = \frac{1}{3}(a + b \sqrt[3]{2} + c \sqrt[3]{2}^2) \in \mathcal{O}_K\) 但 \(\alpha \notin \mathbb{Z}[\sqrt[3]{2}]\),且可假设 \(a, b, c \in \{0, -1, 1\}\)。\(\alpha\) 的最小多项式为:
\[
X^3 - a X^2 + \frac{1}{3}(a^2 - 2bc) X - \frac{1}{27}(a^3 + 2b^3 + 4c^3 - 6abc)
\]
此多项式在此情况下非整,矛盾。故 \(m = 1\),\(\mathcal{O}_K = \mathbb{Z}[\sqrt[3]{2}]\)。  

\textbf{解答2:}
利用艾森斯坦多项式和一引理给出另一证明。

\textbf{艾森斯坦多项式:} 若多项式 \(X^n + a_{n-1} X^{n-1} + \cdots + a_0 \in \mathbb{Z}[X]\) 对素数 \(p\) 满足 \(p \mid a_i\)(\(1 \leq i \leq n-1\))且 \(p \mid a_0\) 但 \(p^2 \nmid a_0\),则为艾森斯坦多项式。  

\textbf{引理1:} 设 \(K\) 为 \(n\) 次数域,\(\alpha \in K\) 为 \(n\) 次代数整数,其最小多项式对素数 \(p\) 为艾森斯坦多项式,则 \(p \nmid \left(\mathcal{O}_K : \mathbb{Z}[\alpha]\right)\)。  

证明:设 \(f(X) = X^n + a_{n-1} X^{n-1} + \cdots + a_0\) 为 \(\alpha\) 的最小多项式。若 \(p \mid \left(\mathcal{O}_K : \mathbb{Z}[\alpha]\right)\),则存在 \(\beta \in \mathcal{O}_K\),使得 \(p \beta \in \mathbb{Z}[\alpha]\) 但 \(\beta \notin \mathbb{Z}[\alpha]\)。写:
\[
p \beta = b_{n-1} \alpha^{n-1} + \cdots + b_1 \alpha + b_0, \quad b_i \in \mathbb{Z}
\]
且不全被 \(p\) 整除。取最小 \(j\)(\(0 \leq j \leq n-1\))使得 \(p \nmid b_j\)。因 \(p \mid b_i\)(\(i < j\)),则 \(\frac{b_i}{p} \alpha^i \in \mathcal{O}_K\)。定义:
\[
\gamma = \frac{b_{n-1}}{p} \alpha^{n-1} + \cdots + \frac{b_j}{p} \alpha^j = \beta - \frac{b_{j-1}}{p} \alpha^{j-1} - \cdots - \frac{b_0}{p} \in \mathcal{O}_K
\]
因 \(f(X)\) 为艾森斯坦多项式,\(\frac{1}{p} \alpha^n = -\frac{a_{n-1}}{p} \alpha^{n-1} - \cdots - \frac{a_0}{p} \in \mathcal{O}_K\)。于是:
\[
\frac{b_j}{p} \alpha^{n-1} = \gamma \alpha^{n-j-1} - \left( b_{n-1} \alpha^{n-j-2} + \cdots + b_{j+1} \right) \frac{1}{p} \alpha^n \in \mathcal{O}_K
\]
计算范数:
\[
N_{K \mid \mathbb{Q}}\left( \frac{b_j}{p} \alpha^{n-1} \right) = \frac{b_j^n}{p^n} N_{K \mid \mathbb{Q}}(\alpha)^{n-1} = \frac{b_j^n a_0^{n-1}}{p^n} \notin \mathbb{Z}
\]
因 \(p \nmid b_j\) 且 \(p^2 \nmid a_0\),矛盾。故 \(p \nmid \left(\mathcal{O}_K : \mathbb{Z}[\alpha]\right)\)。  

结合\textbf{艾森斯坦多项式}和\textbf{引理1}, 设 \(m = \left(\mathcal{O}_K : \mathbb{Z}[\sqrt[3]{2}]\right)\),则 \(m^2 d_K = -108\)。故 \(m = 1, 2, 3\) 或 6。需证 2 和 3 不整除 \(m\)。  
\(\sqrt[3]{2}\) 的最小多项式 \(X^3 - 2\) 对 2 为艾森斯坦多项式,故 \(2 \nmid m\)。设 \(\alpha = 1 + \sqrt[3]{2}\),则 \(K = \mathbb{Q}(\alpha)\)。\(\alpha\) 的最小多项式为 \(X^3 - 3X^2 + 3X - 3\),对 3 为艾森斯坦多项式,故 \(3 \nmid \left(\mathcal{O}_K : \mathbb{Z}[\alpha]\right)\)。因 \(\mathbb{Z}[\alpha] = \mathbb{Z}[\sqrt[3]{2}]\),则 \(3 \nmid m\)。故 \(m = 1\)。  


\item[5.4] \textbf{问题:} 设 \( A \subset B \) 是环的扩展,\( x \) 是 \( B \) 中的元素。证明以下条件等价:
  \subitem (1) \( x \) 在 \( A \) 上是整的;
  \subitem (2) 对于 \( A \) 的每个乘闭子集 \( S \),\( x \in S^{-1} B \) 在 \( S^{-1} A \) 上是整的;
  \subitem (3) 对于 \( A \) 的每个素理想 \( p \),\( x \in B_p \) 在 \( A_p \) 上是整的;
  \subitem (4) 对于 \( A \) 的每个极大理想 \( \mathfrak{m} \),\( x \in B_{\mathfrak{m}} \) 在 \( A_{\mathfrak{m}} \) 上是整的。

\textbf{解答:}此问题刻画了整性的局部性质。 假设条件 (4) 成立,即对于每个极大理想 \( \mathfrak{m} \),\( x \in B_{\mathfrak{m}} \) 在 \( A_{\mathfrak{m}} \) 上是整的,则存在首一多项式 \( f_{\mathfrak{m}}(t) \in A_{\mathfrak{m}}[t] \) 使得 \( f_{\mathfrak{m}}(x) = 0 \)。通过通分,可将 \( f_{\mathfrak{m}}(t) \) 提升为 \( g_{\mathfrak{m}}(t) \in A[t] \),其首项系数 \( a_{\mathfrak{m}} \notin \mathfrak{m} \)。因为所有 \( a_{\mathfrak{m}} \) 共同生成 \( A \)(由极大理想的性质),可通过这些 \( g_{\mathfrak{m}}(t) \) 黏合得到全局首一多项式 \( f(t) \in A[t] \),满足 \( f(x) = 0 \)。故 \( x \) 在 \( A \) 上是整的,即条件 (1) 成立。

由局部化保持整性的性质,可得:条件 (1) \(\Rightarrow\) 条件 (2) \(\Rightarrow\) 条件 (3) \(\Rightarrow\) 条件 (4)。因此,条件 (1) 至 (4) 等价。

\item[5.7] \textbf{问题:} 设 \( A \subset B \) 是环的扩展,\( A' \) 是 \( B \) 中 \( A \) 的子代数。证明以下条件等价:
  \subitem (1) \( A' \) 是 \( A \) 在 \( B \) 中的整闭包;
  \subitem (2) 对于 \( A \) 的每个乘闭子集 \( S \),\( S^{-1} A' \) 是 \( S^{-1} A \) 在 \( S^{-1} B \) 中的整闭包;
  \subitem (3) 对于 \( A \) 的每个素理想 \( p \),\( A'_p \) 是 \( A_p \) 在 \( B_p \) 中的整闭包;
  \subitem (4) 对于 \( A \) 的每个极大理想 \( \mathfrak{m} \),\( A'_{\mathfrak{m}} \) 是 \( A_{\mathfrak{m}} \) 在 \( B_{\mathfrak{m}} \) 中的整闭包。

\textbf{解答:} 
此问题刻画了整闭包的局部性质。
 若条件 (1) 成立,即 \( A' \) 是 \( A \) 在 \( B \) 中的整闭包,则由 5.4,\( S^{-1} A' \subset \) \( S^{-1} A \) 在 \( S^{-1} B \) 中的整闭包。反过来,若 \( \frac{b}{s} \in S^{-1} B \) 在 \( S^{-1} A \) 上整,设其最小多项式为 \( X^n + \frac{a_{n-1}}{s_{n-1}} X^{n-1} + \cdots + \frac{a_0}{s_0} = 0 \)。则 \( s_0 s_1 \cdots s_{n-1} b \) 在 \( A \) 上整,故属于 \( A' \)。因此,\( \frac{b}{s} = \frac{s_0 s_1 \cdots s_{n-1} b}{s_0 s_1 \cdots s_{n-1}} \in S^{-1} A' \),\( S^{-1} A' \) 是整闭包,即条件 (2) 成立。
 由局部化定义,条件 (2) \(\Rightarrow\) 条件 (3) \(\Rightarrow\) 条件 (4)。
 若条件 (4) 成立,即对于每个极大理想 \( \mathfrak{m} \),\( A'_{\mathfrak{m}} \) 是 \( A_{\mathfrak{m}} \) 在 \( B_{\mathfrak{m}} \) 中的整闭包,则 \( A' \) 中元素在 \( B_{\mathfrak{m}} \) 中的像属于 \( A'_{\mathfrak{m}} \),故在 \( A_{\mathfrak{m}} \) 上整,由 5.4,属于 \( A \) 在 \( B \) 中的整闭包。反过来,若 \( b \in B \) 在 \( A \) 上整,则其在 \( B_{\mathfrak{m}} \) 中的像属于 \( A'_{\mathfrak{m}} \)。因 \( \frac{b}{1} \in A'_{\mathfrak{m}} \) 对所有 \( \mathfrak{m} \) 成立,故 \( b \in A' \)。因此,\( A' \) 是整闭包,即条件 (1) 成立。

综上,条件 (1) 至 (4) 等价。

\item[6] 
\textbf{问题:} 证明 \(\left\{1, \theta, \frac{1}{2}(\theta + \theta^2)\right\}\) 是 \(\mathbb{Q}(\theta)\) 的整基,其中 \(\theta^3 - \theta - 4 = 0\)。\\
\textbf{解答:} 因为问题涉及到判别式的计算,并且问题的证明较为复杂,需要到一些中间结论,方可得证。所以首先给出基本定义和一些中间结论,再作完整证明。

\textbf{定义(多项式判别式)}
对于多项式 \(f(X) \in K[X]\),其根为 \(\theta_1, \theta_2, \dots, \theta_n\),判别式为:
\[
\Delta(f) = \prod_{1 \leq i < j \leq n} (\theta_i - \theta_j)^2
\]

\textbf{命题1:} 三次多项式 \( X^3 + aX + b \) 的判别式满足:
\[
\Delta(X^3 + aX + b) = -27b^2 - 4a^3.
\]

\textbf{证明:} 设 \( f(X) = X^3 + aX + b \) 的根为其分裂域中的 \( \theta_1, \theta_2, \theta_3 \)。判别式定义为:
\[
\Delta(f) = \prod_{1 \leqslant i < j \leqslant 3} (\theta_i - \theta_j)^2.
\]
由对称多项式理论,\(\Delta(f)\) 是 \( \theta_1, \theta_2, \theta_3 \) 的齐次对称多项式。根据 Vieta 公式,根的初等对称和为:
\[
s_1 = \theta_1 + \theta_2 + \theta_3 = 0, \quad s_2 = \theta_1 \theta_2 + \theta_1 \theta_3 + \theta_2 \theta_3 = a, \quad s_3 = \theta_1 \theta_2 \theta_3 = -b.
\]
判别式是 6 次齐次多项式,故可设 \(\Delta(f) = v a^3 + w b^2\)。为确定系数 \( v \) 和 \( w \),考虑两个特例:
 取 \( f(X) = X^3 - X \)(\( a = -1, b = 0 \)),根为 \(-1, 0, 1\),则:
  \[
  \Delta(f) = (-1 - 0)^2 (0 - 1)^2 (1 - (-1))^2 = 1 \cdot 1 \cdot 4 = 4 = v (-1)^3 = -v,
  \]
  故 \( v = -4 \)。
  
 取 \( f(X) = X^3 - 1 \)(\( a = 0, b = -1 \)),根为 \( 1, \omega, \omega^2 \)(\(\omega\) 为三次单位根),则:
  \[
  \Delta(f) = (1 - \omega)^2 (\omega - \omega^2)^2 (\omega^2 - 1)^2 = (1 - \omega)^6 = (-3\omega)^3 = -27 = w (-1)^2 = w,
  \]
  故 \( w = -27 \)。
  
因此,\(\Delta(f) = -4 a^3 - 27 b^2 = -27 b^2 - 4 a^3\)。证毕。

\textbf{引理1:} 设 \( f(t) \in A[t_1, t_2, \ldots, t_n] \) 是对称多项式,次数为 \( d \)。存在权不超过 \( d \) 的多项式 \( g(X_1, \ldots, X_n) \),使得:
\[
f(t) = g(s_1, s_2, \ldots, s_n),
\]
其中 \( s_i \) 是 \( t_1, t_2, \ldots, t_n \) 的第 \( i \) 个初等对称多项式。

\textbf{证明:} 设 \( f(t) \in A[t_1, t_2, \ldots, t_n] \) 是次数为 \( d \) 的对称多项式。由对称多项式基本定理,任何对称多项式可表示为初等对称多项式 \( s_1, s_2, \ldots, s_n \) 的多项式,即存在 \( g(X_1, \ldots, X_n) \in A[X_1, \ldots, X_n] \) 使得 \( f(t) = g(s_1, s_2, \ldots, s_n) \)。定义单项式 \( X_1^{v_1} X_2^{v_2} \cdots X_n^{v_n} \) 的权为 \( v_1 + 2 v_2 + \cdots + n v_n \),多项式的权为其单项式的最大权。因为 \( f(t) \) 是齐次且次数为 \( d \),其每一项的权(以 \( t_i \) 的次数加权)恰为 \( d \)。在 \( g \) 中,\( s_i \) 的权为 \( i \),故 \( g \) 的每一项 \( X_1^{v_1} \cdots X_n^{v_n} \) 的权 \( v_1 + 2 v_2 + \cdots + n v_n \leq d \),否则 \( f(t) \) 的次数将超过 \( d \),矛盾。因此,存在权不超过 \( d \) 的 \( g \) 满足要求。

\textbf{命题2:}: 设 \( f(X) = X^n + aX + b \)(\( a, b \in K \))是不可约且可分的,其判别式为:
\[
\Delta(X^n + aX + b) = (-1)^{\frac{n(n-1)}{2}} \left( n^n b^{n-1} + (-1)^{n-1} (n-1)^{n-1} a^n \right).
\]

\textbf{证明:} 设 \( f(X) = X^n + aX + b \) 的根在其分裂域中为 \( \theta_1, \theta_2, \ldots, \theta_n \)。判别式定义为:
\[
\Delta(f) = \prod_{1 \leqslant i < j \leqslant n} (\theta_i - \theta_j)^2.
\]
利用 \( f'(X) = n X^{n-1} + a \),有:
\[
\Delta(f) = (-1)^{\frac{n(n-1)}{2}} \prod_{i=1}^n \prod_{j \neq i} (\theta_i - \theta_j) = (-1)^{\frac{n(n-1)}{2}} \prod_{i=1}^n f'(\theta_i) = (-1)^{\frac{n(n-1)}{2}} N_{L \mid K}(f'(\theta)),
\]
其中 \( \theta \) 是任一根,\( L = K(\theta) \)。令 \( \gamma = f'(\theta) = n \theta^{n-1} + a \),则:
\[
f(X) = (X - \theta) h(X) + f(\theta) = (X - \theta) h(X),
\]
且 \( f\left(-\frac{n b}{\gamma + (n-1) a}\right) = 0 \),故 \( \theta = -\frac{n b}{\gamma + (n-1) a} \),\( K(\gamma) = K(\theta) \)。计算 \( \gamma \) 的最小多项式,设:
\[
f\left(\frac{-n b}{X + (n-1) a}\right) = \frac{P(X)}{Q(X)},
\]
则 \( P(\gamma) = 0 \),且:
\[
P(X) = (X + (n-1) a)^n - n a (X + (n-1) a)^{n-1} + (-n)^n b^{n-1}.
\]
范数为:
\[
N_{L \mid K}(\gamma) = (-1)^n P(0) = (-1)^{n-1} (n-1)^{n-1} a^n + n^n b^{n-1} .
\]
代入得:
\[
\Delta(f) = (-1)^{\frac{n(n-1)}{2}} \left( n^n b^{n-1} + (-1)^{n-1} (n-1)^{n-1} a^n \right).
\]
证毕。

\textbf{命题3:} 设 \(\mathfrak{a} \subset \mathfrak{a}' \) 是数域 \( K \) 中两个非零的有限生成 \(\mathcal{O}_K\)-子模,指数(index) \( (\mathfrak{a}' : \mathfrak{a}) \) 是有限的,并且满足关系:
\[
d(\mathfrak{a}) = (\mathfrak{a}' : \mathfrak{a})^2 d(\mathfrak{a}').
\]


有了以上探索,现在进入正式证明环节。

回看问题,首先计算基 \(\left\{1, \theta, \frac{1}{2}(\theta + \theta^2)\right\}\) 的判别式。设 \(\sigma_1, \sigma_2, \sigma_3\) 为 \(\mathbb{Q}(\theta) \to \mathbb{C}\) 的所有嵌入,\(\theta_i = \sigma_i \theta\)(\(i = 1, 2, 3\))。则:
\[
\begin{aligned}
d\left(1, \theta, \frac{1}{2}(\theta + \theta^2)\right) &= \det\begin{pmatrix}
1 & \theta_1 & \frac{1}{2}(\theta_1 + \theta_1^2) \\
1 & \theta_2 & \frac{1}{2}(\theta_2 + \theta_2^2) \\
1 & \theta_3 & \frac{1}{2}(\theta_3 + \theta_3^2)
\end{pmatrix}^2 \\
&= \frac{1}{4} \det\begin{pmatrix}
1 & \theta_1 & \theta_1^2 \\
1 & \theta_2 & \theta_2^2 \\
1 & \theta_3 & \theta_3^2
\end{pmatrix}^2 \\
&= \frac{1}{4} \prod_{1 \leq i < j \leq 3} (\theta_i - \theta_j)^2
\end{aligned}
\]

根据\textbf{命题1:} \(\Delta(X^3 + aX + b) = -27b^2 - 4a^3\)。  
此处 \(f(X) = X^3 - \theta - 4\),\(a = -1\),\(b = -4\)。则:
\[
\Delta(f) = -27(-4)^2 - 4(-1)^3 = -27 \cdot 16 - 4 \cdot (-1) = -432 + 4 = -428
\]
故:
\[
d\left(1, \theta, \frac{1}{2}(\theta + \theta^2)\right) = \frac{1}{4} \cdot (-428) = -107
\]
因 \(-107\) 为素数,由\textbf{命题3},\(\left(\mathcal{O}_K : \mathbb{Z}[\theta, \frac{1}{2}(\theta + \theta^2)]\right)^2 d_K = -107\)。指数只能为 1,故 \(\left\{1, \theta, \frac{1}{2}(\theta + \theta^2)\right\}\) 为整基。  
由\textbf{命题1}的证明: 设 \(\theta_1, \theta_2, \theta_3\) 为 \(f(X) = X^3 + aX + b\) 的根。\(\Delta(f)\) 为 \(\theta_1, \theta_2, \theta_3\) 的对称多项式。由维塔(Vieta)公式,初等对称多项式为 \(s_1 = 0\)、\(s_2 = a\)、\(s_3 = -b\)。由\textbf{引理1},\(\Delta(f) = v a^3 + w b^2\)。  
- 取 \(f(X) = X^3 - X\)(\(a = -1, b = 0\)),根为 \(-1, 0, 1\),\(\Delta(f) = 4 = -v\),故 \(v = -4\)。  
- 取 \(f(X) = X^3 - 1\)(\(a = 0, b = -1\)),根为三次单位根,\(\Delta(f) = -27 = w\),故 \(w = -27\)。  
因此,\(\Delta(f) = -27 b^2 - 4 a^3\)。  



\item[7] 
\textbf{问题:}(Stickelberger 判别式关系)证明代数域 \(K\) 的判别式 \(d_K\) 总是 \(\equiv 0 \text{ 或 } 1 \pmod{4}\)。\\
\textbf{解答:} 设 \(\alpha_1, \dots, \alpha_m\) 为 \(\mathcal{O}_K\) 的整基,\(\sigma_1, \dots, \sigma_m\) 为 \(K\) 的所有嵌入。判别式为:
\[
d_K = \det(\sigma_i \alpha_j)^2
\]
行列式 \(\det(\sigma_i \alpha_j)\) 为所有嵌入作用于 \(\alpha_1, \dots, \alpha_m\) 的排列乘积之和。设 \(P\) 为偶排列项之和,\(-N\) 为奇排列项之和,则:
\[
d_K = (P - N)^2 = (P + N)^2 - 4 P N
\]
设 \(G\) 为 \(K\) 在 \(\mathbb{Q}\) 上伽罗瓦闭包的伽罗瓦群。每个嵌入可延拓为 \(G\) 中的元素,反之亦然。对任意 \(\tau \in G\),\(\tau \sigma_1, \dots, \tau \sigma_m\) 为 \(\sigma_1, \dots, \sigma_m\) 的一个排列。根据排列的奇偶性:
- 若为偶排列,则 \(\tau P = P, \tau N = N\);
- 若为奇排列,则 \(\tau P = N, \tau N = P\)。  
故 \(P + N\) 和 \(P N\) 被 \(G\) 固定,在 \(\mathbb{Q}\) 中。因其对 \(\mathbb{Z}\) 整,必为整数。因此:
\[
d_K \equiv (P + N)^2 \equiv 0 \text{ 或 } 1 \pmod{4}
\]

\end{enumerate}


\section{第三章:理想}

\subsection{练习题}

\begin{enumerate}

\begin{enumerate}

\item[1] 
\textbf{问题:} 将 \(33 + 11 \sqrt{-7}\) 分解为 \(\mathbb{Q}(\sqrt{-7})\) 中不可约的整元素。\\
\textbf{解答:} 关于有理数域的二次扩域的整环,有如下命题,可做进一步分析和计算。\\
\textbf{命题1:}设 \(D\) 为不等于 0 或 1 的无平方因子有理整数,\(K = \mathbb{Q}(\sqrt{D})\) 为二次数域,\(d\) 为其判别式。则:
\[
\begin{cases}
d = D, & \text{若 } D \equiv 1 \pmod{4}, \\
d = 4D, & \text{若 } D \equiv 2 \text{ 或 } 3 \pmod{4},
\end{cases}
\]
且 \(K\) 的整基在第二种情况下为 \(\{1, \sqrt{D}\}\),第一种情况下为 \(\left\{1, \frac{1}{2}(1 + \sqrt{D})\right\}\),且在两种情况下均为 \(\left\{1, \frac{1}{2}(d + \sqrt{d})\right\}\)。\\

根据\textbf{命题1},\(\mathbb{Q}(\sqrt{-7})\) 的整数环为 \(\mathbb{Z}\left[\frac{1 + \sqrt{-7}}{2}\right]\)。在此环中,首先分解 \(33 + 11 \sqrt{-7}\) 为:
\[
33 + 11 \sqrt{-7} = 11 \cdot 2 \cdot \frac{3 + \sqrt{-7}}{2}
\]
在 \(\mathbb{Z}\left[\frac{1 + \sqrt{-7}}{2}\right]\) 中,元素的范数为:
\[
N\left(x + y \left(\frac{1 + \sqrt{-7}}{2}\right)\right) = \left(x + \frac{y}{2}\right)^2 + 7 \left(\frac{y}{2}\right)^2
\]
分步骤证明如下:\\
\textbf{步骤 1:} 分解 11 为不可约元素。  
\(N(11) = 121 = 11 \cdot 11\)。11 可能本身不可约,或分解为两个范数为 11 的不可约元素 \(\alpha\) 和 \(\beta\)。考虑方程:
\[
\left(x + \frac{y}{2}\right)^2 + 7 \left(\frac{y}{2}\right)^2 = 11
\]
整数解为 \((1, 2), (-3, 2), (3, -2), (-1, -2)\),对应元素 \(2 + \sqrt{-7}, -2 + \sqrt{-7}, 2 - \sqrt{-7}, -2 - \sqrt{-7}\)。这些元素仅符号或共轭不同,故分解为:
\[
11 = (2 + \sqrt{-7}) \cdot (2 - \sqrt{-7})
\]
此分解在单位因子下唯一。\\
\textbf{步骤 2:} 分解 2 和 \(\frac{3 + \sqrt{-7}}{2}\) 为不可约元素。  
\(N(2) = N\left(\frac{3 + \sqrt{-7}}{2}\right) = 4 = 2 \cdot 2\)。考虑方程:
\[
\left(x + \frac{y}{2}\right)^2 + 7 \left(\frac{y}{2}\right)^2 = 2
\]
整数解为 \((0, 1), (-1, 1), (0, -1), (1, -1)\),对应元素 \(\frac{1 + \sqrt{-7}}{2}, \frac{-1 + \sqrt{-7}}{2}, \frac{-1 - \sqrt{-7}}{2}, \frac{1 - \sqrt{-7}}{2}\)。这些元素仅符号或共轭不同,故分解为:
\[
2 = \frac{1 + \sqrt{-7}}{2} \cdot \frac{1 - \sqrt{-7}}{2}, \quad \frac{3 + \sqrt{-7}}{2} = -\left(\frac{1 - \sqrt{-7}}{2}\right)^2
\]
此分解在单位因子下唯一。\\
\textbf{步骤 3:} 综合分解 \(33 + 11 \sqrt{-7}\):
\[
33 + 11 \sqrt{-7} = -(2 + \sqrt{-7}) \cdot (2 - \sqrt{-7}) \cdot \frac{1 + \sqrt{-7}}{2} \cdot \left(\frac{1 - \sqrt{-7}}{2}\right)^3
\]

\item[2] 
\textbf{问题:} 证明:
\[
54 = 2 \cdot 3^3 = \frac{13 + \sqrt{-47}}{2} \cdot \frac{13 - \sqrt{-47}}{2}
\]
是 \(\mathbb{Q}(\sqrt{-47})\) 中本质不同的两种不可约整元素分解。\\
\textbf{解答:} 由于 \(\frac{13 \pm \sqrt{-47}}{2 \cdot 2}\) 和 \(\frac{13 \pm \sqrt{-47}}{2 \cdot 3}\) 不属于 \(\mathbb{Q}(\sqrt{-47})\) 的整数环,2 和 3(以及 \(2 \cdot 3^3\) 的其他非平凡因子)与 \(\frac{13 + \sqrt{-47}}{2}\) 或 \(\frac{13 - \sqrt{-47}}{2}\) 不关联。因此,这两种分解本质不同。\\
\textbf{备注 2.1:} 54 的分解不止这两种,例如 \(54 = 3^2 \cdot \frac{5 + \sqrt{-47}}{2} \cdot \frac{5 - \sqrt{-47}}{2}\) 为另一种分解。

\item[3] 
\textbf{问题:} 设 \(d\) 为无平方因子整数,\(p\) 为不整除 \(2d\) 的素数,\(\mathcal{O}\) 为 \(\mathbb{Q}(\sqrt{d})\) 的整数环。证明 \((p) = p \mathcal{O}\) 是 \(\mathcal{O}\) 的素理想当且仅当同余式 \(x^2 \equiv d \pmod{p}\) 无解。\\
\textbf{解答:} \\
\textbf{命题1:}设 \(D\) 为不等于 0 或 1 的无平方因子有理整数,\(K = \mathbb{Q}(\sqrt{D})\) 为二次数域,\(d\) 为其判别式。则:
\[
\begin{cases}
d = D, & \text{若 } D \equiv 1 \pmod{4}, \\
d = 4D, & \text{若 } D \equiv 2 \text{ 或 } 3 \pmod{4},
\end{cases}
\]
且 \(K\) 的整基在第二种情况下为 \(\{1, \sqrt{D}\}\),第一种情况下为 \(\left\{1, \frac{1}{2}(1 + \sqrt{D})\right\}\),且在两种情况下均为 \(\left\{1, \frac{1}{2}(d + \sqrt{d})\right\}\)。\\

由\textbf{命题1:},\(\mathcal{O}\) 为 \(\mathbb{Z}[\sqrt{d}]\) 或 \(\mathbb{Z}\left[\frac{1 + \sqrt{d}}{2}\right]\)。  
若 \(d\) 是模 \(p\) 的二次剩余,即存在整数 \(x\) 使得 \(x^2 \equiv d \pmod{p}\)。则 \(p \mid x^2 - d = (x + \sqrt{d})(x - \sqrt{d})\)。但因 \(p \neq 2\),\(\frac{x + \sqrt{d}}{p}\) 和 \(\frac{x - \sqrt{d}}{p}\) 不在 \(\mathcal{O}\) 中,故 \(p\) 不是素元素,\((p)\) 不是素理想。  
反之,若 \(d\) 不是模 \(p\) 的二次剩余,证明 \(p\) 是素元素。设 \(\mathcal{O}\) 中元素 \(x_1 + y_1 \sqrt{d}\) 和 \(x_2 + y_2 \sqrt{d}\) 的乘积在 \((p)\) 中。因 \(p \neq 2\),\(\frac{x}{p} \in \mathbb{Z}\) 与 \(\frac{x}{p} \in \frac{1}{2} \mathbb{Z}\) 等价,可归约至 \(\mathcal{O} = \mathbb{Z}[\sqrt{d}]\)。由:
\[
(x_1 + y_1 \sqrt{d})(x_2 + y_2 \sqrt{d}) \in (p)
\]
得:
\[
p^2 = N(p) \mid N(x_1 + y_1 \sqrt{d}) N(x_2 + y_2 \sqrt{d})
\]
则 \(p\) 整除 \(N(x_1 + y_1 \sqrt{d})\) 或 \(N(x_2 + y_2 \sqrt{d})\),例如前者:
\[
p \mid N(x_1 + y_1 \sqrt{d}) = x_1^2 - d y_1^2
\]
即:
\[
x_1^2 \equiv d y_1^2 \pmod{p}
\]
因 \(d\) 不是模 \(p\) 的二次剩余,\(y_1\) 在模 \(p\) 下不可逆,故 \(p \mid y_1\) 且 \(p \mid x_1\)。因此 \(x_1 + y_1 \sqrt{d} \in (p)\),\((p)\) 是素理想。

\item[4]
\textbf{问题:} 具有有限个素理想的戴德金域是主理想域。\\
\textbf{解答:} 解答此题需要如下基本定理。\\
\textbf{定理1:} 环 \(\mathcal{O}\) 中每个不同于 (0) 和 (1) 的理想  $\mathfrak{a}$ 都可以分解为
$$
\mathfrak{a}=\mathfrak{p}_1 \cdots \mathfrak{p}_r
$$
其中 $\mathfrak{p}_i$ 是 $\mathcal{O}$ 中的非零素理想,且这种分解除了因子的顺序外是唯一的。

首先证明 \(\mathcal{O}\) 仅有一个非零素理想 \(\mathfrak{p}\) 的情况(例如局部环)。存在 \(\pi \in \mathfrak{p} \setminus \mathfrak{p}^2\)。考虑理想 \((\pi)\),由\textbf{定理1:},其唯一分解为 \((\pi) = \mathfrak{p}^\nu\)。但 \(\pi \notin \mathfrak{p}^2\),故 \((\pi) = \mathfrak{p}\),表明 \(\mathcal{O}\) 唯一素理想为主理想。由\textbf{定理1:},所有理想为主理想。  
对于 \(\mathcal{O}\) 有有限个素理想 \(\mathfrak{p}_1, \dots, \mathfrak{p}_r\) 的情况,若 \(\mathfrak{a} = \mathfrak{p}_1^{\nu_1} \cdots \mathfrak{p}_r^{\nu_r} \neq 0\),取 \(\pi_i \in \mathfrak{p}_i \setminus \mathfrak{p}_i^2\)。由中国剩余定理,存在 \(a \in \mathcal{O}\) 对应余类 \(\pi_i^{\nu_i} \pmod{\mathfrak{p}_i^{\nu_i + 1}}\)。设 \((a) = \mathfrak{p}_1^{\mu_1} \cdots \mathfrak{p}_r^{\mu_r}\)。因 \(a \equiv \pi_i^{\nu_i} \pmod{\mathfrak{p}_i^{\nu_i + 1}}\),\(a \notin \mathfrak{p}_i^{\nu_i + 1}\),故 \(\mu_i \leq \nu_i\);且 \(a \in \mathfrak{p}_i^{\mu_i}\),故 \(\mu_i \geq \nu_i\)。因此 \((a) = \mathfrak{a}\)。

\item[5]
\textbf{问题:} 戴德金域 \(\mathcal{O}\) 被非零理想 \(\mathfrak{a}\) 商得的商环 \(\mathcal{O} / \mathfrak{a}\) 是主理想域。\\
\textbf{解答:} 此结论错误,\(\mathcal{O} / \mathfrak{a}\) 一般不是整环。正确结论为:\(\mathcal{O} / \mathfrak{a}\) 是主环,即每个理想为主理想。)\\
首先证明 \(\mathfrak{a} = \mathfrak{p}^n\) 的情况。\(\mathcal{O} / \mathfrak{a}\) 的唯一真理想为 \(\mathfrak{p} / \mathfrak{p}^n, \dots, \mathfrak{p}^{n-1} / \mathfrak{p}^n\)。取 \(\pi \in \mathfrak{p} \setminus \mathfrak{p}^2\),则 \(\pi^i \mathcal{O} / \mathfrak{p}^n = \mathfrak{p}^i / \mathfrak{p}^n\)(因 \(\pi^i \in \mathfrak{p}^i \setminus \mathfrak{p}^{i+1}\)),故所有真理想为主理想。  
一般情况,注意到 PID 的商环仍是主环,用归纳法即可。\\
\textbf{问题:} 设 \(D\) 为整环,则以下等价:\\
1. \(D\) 是戴德金域;\\
2. \(D\) 是诺特环,且对每个乘法闭子集 \(S\),\(S^{-1} D\) 是戴德金域;\\
3. \(D\) 是诺特环,且对每个素理想 \(\mathfrak{p}\),\(D_\mathfrak{p}\) 是戴德金域;\\
4. \(D\) 是诺特环,且对每个极大理想 \(\mathfrak{m}\),\(D_\mathfrak{m}\) 是戴德金域。\\
\textbf{解答:} 以上刻画说明戴德金性是局部性质。证明如下:
戴德金域是诺特、整闭且非零素理想极大的环。\(S^{-1} D\) 的素理想对应于 \(D \setminus S\) 中的素理想,故“极大”条件是局部性质。由整闭性和诺特性的局部性得证。\\
\textbf{备注} “戴德金性是局部性质”指上述结论,是非严格局部性质。例如,\(\mathbb{Z}\) 在 \(\mathbb{Q}\) 加入所有素数 \(p\) 的 \(p\) 次单位根的域中整闭包非诺特,故非戴德金域。

\item[6]
\textbf{问题:} 戴德金域的每个理想可由两个元素生成。\\
\textbf{解答:} 对 \(\mathcal{O}\) 的每个理想 \(\mathfrak{a}\),取 \(a \in \mathfrak{a}\),商环 \(\mathcal{O} / (a)\) 为主环。\(\mathfrak{a}\) 在 \(\mathcal{O} / (a)\) 中的像是主理想,由 \(b \pmod{(a)}\)(\(b \in \mathfrak{a}\))生成,故 \(\mathfrak{a} = (a) + (b)\)。\\
\textbf{问题:} 戴德金域 \(\mathcal{O}\) 是 PID 当且仅当其类群平凡。\\
\textbf{解答:}若类群平凡,每分式理想为主理想,故 \(\mathcal{O}\) 是 PID。反之,若 \(\mathcal{O}\) 是 PID,对每个分式理想 \(\mathfrak{a}\),存在 \(c \in \mathcal{O}\) 使 \(c \mathfrak{a}\) 为主理想,故 \(\mathfrak{a}\) 为主理想,类群平凡。

% \item[7]

% \textbf{问题:} 在诺特环 \(R\) 中,若每个素理想极大,则每个理想降链 \(a_1 \supseteq a_2 \supseteq \cdots\) 最终稳定。\\
% \textbf{备注} “每个素理想极大”意味着 \((0)\) 不能是素理想,除非 \(R\) 是域。

% \textbf{解答:} 
% 正式解答之前,需要回顾如下引理:\\

% \textbf{ 引理7.1} 对于环 \(\mathcal{O}\) 中每个非零理想 \(\mathfrak{a} \neq 0\),存在非零素理想 \(\mathfrak{p}_1, \mathfrak{p}_2, \ldots, \mathfrak{p}_r\),使得
% \[
% \mathfrak{a} \supseteq \mathfrak{p}_1 \mathfrak{p}_2 \cdots \mathfrak{p}_r
% \]

% \textbf{引理证明:} 假设那些不满足该条件的理想集合 \(\mathfrak{M}\) 非空。由于 \(\mathcal{O}\) 是诺特环(Noetherian),每个理想的升链都会稳定。因此,\(\mathfrak{M}\) 关于包含关系是归纳序的,从而存在一个极大元素 \(\mathfrak{a}\)。这个 \(\mathfrak{a}\) 不可能是素理想,因此存在元素 \(b_1, b_2 \in \mathcal{O}\),使得 \(b_1 b_2 \in \mathfrak{a}\),但 \(b_1, b_2 \notin \mathfrak{a}\)。设 \(\mathfrak{a}_1 = (b_1) + \mathfrak{a}\),\(\mathfrak{a}_2 = (b_2) + \mathfrak{a}\)。则有 \(\mathfrak{a} \varsubsetneqq \mathfrak{a}_1\),\(\mathfrak{a} \varsubsetneqq \mathfrak{a}_2\),并且 \(\mathfrak{a}_1 \mathfrak{a}_2 \subseteq \mathfrak{a}\)。由 \(\mathfrak{a}\) 的极大性,\(\mathfrak{a}_1\) 和 \(\mathfrak{a}_2\) 都包含一些素理想的乘积,而这些乘积的乘积又包含于 \(\mathfrak{a}\) 中,这导致矛盾。

% 现对问题做解答:
% 首先,我们证明在诺特环中,\((0)\) 具有素分解 \((0) = \mathfrak{p}_1 \cdots \mathfrak{p}_r\)。实际上,由\textbf{ 引理7.1},在诺特环中,每个真理想(包括 \((0)\)) 具有素分解。因此,我们有 \((0) \supset \mathfrak{p}_1 \cdots \mathfrak{p}_r\)。但 \((0) = \{0\}\),故 \((0) = \mathfrak{p}_1 \cdots \mathfrak{p}_r\)。  
% 此外,\(\mathfrak{p}_1, \dots, \mathfrak{p}_r\) 是所有素理想。若不是,则存在另一个素理想 \(\mathfrak{q}\) 使得 \(\mathfrak{p}_1 \cdots \mathfrak{p}_r = (0) \subset \mathfrak{q}\)。因此,其中某个素理想,例如 \(\mathfrak{p}_1\),必须被包含在 \(\mathfrak{q}\) 中。但由于每个素理想是极大的,故 \(\mathfrak{p}_1 = \mathfrak{q}\),这产生矛盾。  
% 因此,我们得到一个理想的降链:
% \[
% R \supset \mathfrak{p}_1 \supset \mathfrak{p}_1 \mathfrak{p}_2 \supset \cdots \supset \mathfrak{p}_1 \cdots \mathfrak{p}_r = (0)
% \]
% 每个因子 \(\mathfrak{p}_1 \cdots \mathfrak{p}_{i-1} / \mathfrak{p}_1 \cdots \mathfrak{p}_i\) 是域 \(R / \mathfrak{p}_i\) 上的向量空间。对于向量空间 \(V\),链条件等价于 \(\dim V\) 有限。因此,\(\mathfrak{p}_1 \cdots \mathfrak{p}_{i-1} / \mathfrak{p}_1 \cdots \mathfrak{p}_i\) 是诺特模当且仅当它是阿廷模,作为 \(R / \mathfrak{p}_i\)-模(进而作为 \(R\)-模)。反复应用如下\textbf{引理 7.2:},我们看到 \(R\) 是诺特环当且仅当它是阿廷环。

% \textbf{引理 7.2:} 设 \(0 \to M' \to M \to M'' \to 0\) 为 \(R\)-模的短正合序列,则 \(M\) 是诺特(或阿廷)模当且仅当 \(M'\) 和 \(M''\) 都是诺特(或阿廷)模。\\
% 证明:\\
% 对于 “当且” 部分, 注意到 \(M\) 的子模链 \((M_i)\) 由其在 \(M'\) 中的逆像 \((M_i')\) 和在 \(M''\) 中的像 \((M_i'')\) 控制。(Five-Lemma):考虑以下交换图:
% \[
% \begin{tikzcd}
% 0 \arrow[r] & M_i' \arrow[r] \arrow[d, "f_i'"] & M_i \arrow[r] \arrow[d, "f_i"] & M_i'' \arrow[r] \arrow[d, "f_i''"] & 0 \\
% 0 \arrow[r] & M_{i+1}' \arrow[r] & M_{i+1} \arrow[r] & M_{i+1}'' \arrow[r] & 0
% \end{tikzcd}
% \]
% 事实上,对于足够大的 \(i\),嵌入 \(f_i': M_i' \to M'\) 和 \(f_i'': M_i'' \to M''\) 成为恒等映射,因此Five-Lemma,\(f_i: M_i \to M\) 也为恒等映射。因此,\(M\) 的子模链最终稳定。\\
% 对于 “仅当” 部分, 仅仅注意到 \(M'\) 或 \(M''\) 的子模链会诱导 \(M\) 的子模链,故因 \(M\) 是诺特(或阿廷)模,\(M'\) 和 \(M''\) 的链也稳定。

% \textbf{7.3(合成列):} \(R\)-模 \(M\) 的合成列是子模链:
% \[
% M = M_0 \supset M_1 \supset \cdots \supset M_n = 0
% \]
% 其中无法插入额外子模。若 \(M\) 有合成列,则所有合成列长度相同(称为 \(M\) 的长度),任何子模链可扩展为合成。\\
% \textbf{命题 7.4:} \(M\) 有合成列当且仅当 \(M\) 既是诺特模又是阿廷模。\\
% 证明:若 \(M\) 有合成列,子模链长度有界,故 \(M\) 是诺特和阿廷模。  
% 若 \(M\) 是诺特和阿廷模,构造合成列:因 \(M_0 = M\) 是诺特模,存在极大子模 \(M_1\)。类似地,\(M_1\) 有极大子模 \(M_2\)。继续得降链 \(M = M_0 \supset M_1 \supset M_2 \supset \cdots\)。因 \(M\) 是阿廷模,链有限,构成合成列。\\
% \textbf{推论 7.5:} 对向量空间 \(V\),以下等价:\\
% 1. \(V\) 有限维;\\
% 2. \(V\) 长度有限;\\
% 3. \(V\) 是诺特模;\\
% 4. \(V\) 是阿廷模。\\
% 应用此结果,\(\mathfrak{p}_1 \cdots \mathfrak{p}_{i-1} / \mathfrak{p}_1 \cdots \mathfrak{p}_i\) 有限维,其子空间与 \(\mathfrak{p}_1 \cdots \mathfrak{p}_i\) 和 \(\mathfrak{p}_1 \cdots \mathfrak{p}_{i-1}\) 间的理想一一对应,故链 \(R \supset \mathfrak{p}_1 \supset \mathfrak{p}_1 \mathfrak{p}_2 \supset \cdots \supset \mathfrak{p}_1 \cdots \mathfrak{p}_r = (0)\) 可扩展为合成列,\(R\) 是阿廷环,降链稳定。

\item[7]

\textbf{问题:} 在诺特环 \(R\) 中,若每个素理想极大,则每个理想降链 \(a_1 \supseteq a_2 \supseteq \cdots\) 最终稳定。\\
\textbf{备注} "每个素理想极大"意味着 \((0)\) 不能是素理想,除非 \(R\) 是域。

\textbf{解答:} 
我们的证明策略是首先证明这样的环具有特殊的理想结构,然后利用这种结构来证明它同时满足诺特性和阿廷性,从而得出理想降链最终稳定的结论。首先需要以下引理:\\

\textbf{引理7.1} 对于环 \(\mathcal{O}\) 中每个非零理想 \(\mathfrak{a} \neq 0\),存在非零素理想 \(\mathfrak{p}_1, \mathfrak{p}_2, \ldots, \mathfrak{p}_r\),使得
\[
\mathfrak{a} \supseteq \mathfrak{p}_1 \mathfrak{p}_2 \cdots \mathfrak{p}_r
\]

\textbf{引理证明:} 假设那些不满足该条件的理想集合 \(\mathfrak{M}\) 非空。由于 \(\mathcal{O}\) 是诺特环(Noetherian),每个理想的升链都会稳定。因此,\(\mathfrak{M}\) 关于包含关系是归纳序的,从而存在一个极大元素 \(\mathfrak{a}\)。这个 \(\mathfrak{a}\) 不可能是素理想,因此存在元素 \(b_1, b_2 \in \mathcal{O}\),使得 \(b_1 b_2 \in \mathfrak{a}\),但 \(b_1, b_2 \notin \mathfrak{a}\)。设 \(\mathfrak{a}_1 = (b_1) + \mathfrak{a}\),\(\mathfrak{a}_2 = (b_2) + \mathfrak{a}\)。则有 \(\mathfrak{a} \varsubsetneqq \mathfrak{a}_1\),\(\mathfrak{a} \varsubsetneqq \mathfrak{a}_2\),并且 \(\mathfrak{a}_1 \mathfrak{a}_2 \subseteq \mathfrak{a}\)。由 \(\mathfrak{a}\) 的极大性,\(\mathfrak{a}_1\) 和 \(\mathfrak{a}_2\) 都包含一些素理想的乘积,而这些乘积的乘积又包含于 \(\mathfrak{a}\) 中,这导致矛盾。

有了这个引理作为基础,我们现在可以开始解答原问题。我们的证明分为几个关键步骤:首先确定\((0)\)理想的素分解结构,然后分析这种结构所导致的代数性质。

首先,我们证明在诺特环中,\((0)\) 具有素分解 \((0) = \mathfrak{p}_1 \cdots \mathfrak{p}_r\)。实际上,由\textbf{引理7.1},在诺特环中,每个真理想(包括 \((0)\)) 具有素分解。因此,我们有 \((0) \supset \mathfrak{p}_1 \cdots \mathfrak{p}_r\)。但 \((0) = \{0\}\),故 \((0) = \mathfrak{p}_1 \cdots \mathfrak{p}_r\)。

此外,\(\mathfrak{p}_1, \dots, \mathfrak{p}_r\) 是所有素理想。若不是,则存在另一个素理想 \(\mathfrak{q}\) 使得 \(\mathfrak{p}_1 \cdots \mathfrak{p}_r = (0) \subset \mathfrak{q}\)。因此,其中某个素理想,例如 \(\mathfrak{p}_1\),必须被包含在 \(\mathfrak{q}\) 中。但由于每个素理想是极大的,故 \(\mathfrak{p}_1 = \mathfrak{q}\),这产生矛盾。

接下来,我们需要分析环$R$的模结构。我们观察到前面得到的理想链实际上揭示了$R$的层次结构,每一层都可以视为向量空间。具体地,我们得到一个理想的降链:
\[
R \supset \mathfrak{p}_1 \supset \mathfrak{p}_1 \mathfrak{p}_2 \supset \cdots \supset \mathfrak{p}_1 \cdots \mathfrak{p}_r = (0)
\]

每个因子 \(\mathfrak{p}_1 \cdots \mathfrak{p}_{i-1} / \mathfrak{p}_1 \cdots \mathfrak{p}_i\) 是域 \(R / \mathfrak{p}_i\) 上的向量空间。对于向量空间 \(V\),链条件等价于 \(\dim V\) 有限。

为了证明$R$同时具有诺特性和阿廷性,我们需要分析这些层次之间的关系。以下引理将帮助我们建立这种联系:

\textbf{引理 7.2:} 设 \(0 \to M' \to M \to M'' \to 0\) 为 \(R\)-模的短正合序列,则 \(M\) 是诺特(或阿廷)模当且仅当 \(M'\) 和 \(M''\) 都是诺特(或阿廷)模。\\
\textbf{证明:}\\
对于 "当且" 部分, 注意到 \(M\) 的子模链 \((M_i)\) 由其在 \(M'\) 中的逆像 \((M_i')\) 和在 \(M''\) 中的像 \((M_i'')\) 控制。(Five-Lemma):考虑以下交换图:
\[
\begin{tikzcd}
0 \arrow[r] & M_i' \arrow[r] \arrow[d, "f_i'"] & M_i \arrow[r] \arrow[d, "f_i"] & M_i'' \arrow[r] \arrow[d, "f_i''"] & 0 \\
0 \arrow[r] & M_{i+1}' \arrow[r] & M_{i+1} \arrow[r] & M_{i+1}'' \arrow[r] & 0
\end{tikzcd}
\]
事实上,对于足够大的 \(i\),嵌入 \(f_i': M_i' \to M'\) 和 \(f_i'': M_i'' \to M''\) 成为恒等映射,因此根据Five-Lemma,\(f_i: M_i \to M\) 也为恒等映射。因此,\(M\) 的子模链最终稳定。\\
对于 "仅当" 部分, 仅仅注意到 \(M'\) 或 \(M''\) 的子模链会诱导 \(M\) 的子模链,故因 \(M\) 是诺特(或阿廷)模,\(M'\) 和 \(M''\) 的链也稳定。

有了这个引理,我们可以通过归纳法将问题归约为分析各个因子空间的性质。为此,我们需要引入合成列的概念:

\textbf{7.3(合成列):} \(R\)-模 \(M\) 的合成列是子模链:
\[
M = M_0 \supset M_1 \supset \cdots \supset M_n = 0
\]
其中无法插入额外子模。若 \(M\) 有合成列,则所有合成列长度相同(称为 \(M\) 的长度),任何子模链可扩展为合成。\\

\textbf{命题 7.4:} \(M\) 有合成列当且仅当 \(M\) 既是诺特模又是阿廷模。\\
\textbf{证明:} 若 \(M\) 有合成列,子模链长度有界,故 \(M\) 是诺特和阿廷模。若 \(M\) 是诺特和阿廷模,构造合成列:因 \(M_0 = M\) 是诺特模,存在极大子模 \(M_1\)。类似地,\(M_1\) 有极大子模 \(M_2\)。继续得降链 \(M = M_0 \supset M_1 \supset M_2 \supset \cdots\)。因 \(M\) 是阿廷模,链有限,构成合成列。

\textbf{推论 7.5:} 对向量空间 \(V\),以下等价:\\
1. \(V\) 有限维;\\
2. \(V\) 长度有限;\\
3. \(V\) 是诺特模;\\
4. \(V\) 是阿廷模。\\

现在我们已经建立了必要的理论基础,可以将它应用到我们的具体问题上:

将以上结果综合应用到我们的环$R$上,我们发现:各因子 \(\mathfrak{p}_1 \cdots \mathfrak{p}_{i-1} / \mathfrak{p}_1 \cdots \mathfrak{p}_i\) 作为\(R/\mathfrak{p}_i\)上的向量空间是有限维的,其子空间与 \(\mathfrak{p}_1 \cdots \mathfrak{p}_i\) 和 \(\mathfrak{p}_1 \cdots \mathfrak{p}_{i-1}\) 间的理想一一对应。根据引理7.2,反复应用可知,我们的理想链 \(R \supset \mathfrak{p}_1 \supset \mathfrak{p}_1 \mathfrak{p}_2 \supset \cdots \supset \mathfrak{p}_1 \cdots \mathfrak{p}_r = (0)\) 可扩展为合成列,因此$R$不仅是诺特环,而且是阿廷环。

总结上述分析,我们可以得出:在诺特环$R$中,若每个素理想都是极大的,则$R$同时是阿廷环,从而每个理想降链最终稳定。这完成了问题7的证明。

\end{enumerate}

\section{第四章:格}

\subsection{练习题}
\begin{enumerate}

\item[1] 
\textbf{问题:} 证明在 \(\mathbb{R}^n\) 中的格 \(\Gamma\) 是完备的当且仅当商空间 \(\mathbb{R}^n / \Gamma\) 是紧的。\\
\textbf{解答:} 若 \(\Gamma\) 是完备的,则基本网格 \(\Phi\) 满足 \(\Phi + \Gamma = \mathbb{R}^n\)。投影 \(\pi: \mathbb{R}^n \to \mathbb{R}^n / \Gamma\) 将紧集映射为紧集,且 \(\Phi\) 和 \(\Phi + \Gamma\) 在此投影下的像相同,因此 \(\mathbb{R}^n / \Gamma\) 是紧的。  
反之,若 \(\Gamma\) 不完备,设 \(V_0\) 为 \(\Gamma\) 生成的子空间 \(\mathbb{R}^n\) 中的子空间。则存在 \(v \in \mathbb{R}^n \setminus V_0\),故 \(\pi|_{V_0}\) 是单射。因 \(\pi\) 是拓扑群的商映射,故是开映射,\(\pi|_{V_0}\) 是开嵌入。因此 \(\mathbb{R}^n / \Gamma\) 不是紧的。

\item[2] 
\textbf{问题:} 通过构造一个中心对称的凸集 \(X \subset V\) 的例子,证明闵可夫斯基格点定理无法改进,其中 \(\operatorname{vol}(X) = 2^n \operatorname{vol}(\Gamma)\) 但 \(X\) 不含 \(\Gamma\) 的任何非零格点。然而,若 \(X\) 是紧的,则在等式情况下定理 (闵可夫斯基格点) 仍然成立。\\
\textbf{解答:} 
首先回顾闵可夫斯基格点定理\\
\textbf{ 定理(闵可夫斯基格点):} 设 $\Gamma$ 是欧几里得向量空间 $V$ 中的完整格,$X$ 是 $V$ 中的中心对称凸子集。假设
$$
\operatorname{vol}(X)>2^n \operatorname{vol}(\Gamma)
$$
则 $X$ 至少包含一个非零格点 $\gamma \in \Gamma$。

例如,考虑 \(\mathbb{R}^2\) 中的格 \(\mathbb{Z}^2\)。集合 \(X = (-1, 1) \times (-1, 1)\) 是中心对称的凸集,且体积为 4。但 \(X\) 不含任何非零格点。  
对于第二部分,使用定理 (闵可夫斯基格点定理) 的方法。需证明存在两个不同格点 \(\gamma_1, \gamma_2 \in \Gamma\),使得:
\[
\left(\frac{1}{2} X + \gamma_1\right) \cap \left(\frac{1}{2} X + \gamma_2\right) \neq \varnothing
\]
若成立,则 \(\gamma_1 - \gamma_2 \in X \cap \Gamma\)。  
当 \(X\) 是紧的,若 \(\frac{1}{2} X + \gamma\) 两两不相交,则:
\[
\operatorname{vol}(\Phi) > \sum_{\gamma \in \Gamma} \operatorname{vol}\left(\Phi \cap \left(\frac{1}{2} X + \gamma\right)\right)
\]
因 \((\Phi - \gamma) \cap \frac{1}{2} X\) 的体积与 \(\Phi \cap \left(\frac{1}{2} X + \gamma\right)\) 相同,且 \(\Phi - \gamma\)(\(\gamma \in \Gamma\))覆盖整个空间 \(V\),有:
\[
\operatorname{vol}(\Phi) > \sum_{\gamma \in \Gamma} \operatorname{vol}\left((\Phi - \gamma) \cap \frac{1}{2} X\right) = \operatorname{vol}\left(\frac{1}{2} X\right) = \frac{1}{2^n} \operatorname{vol}(X)
\]
这与假设矛盾。

\item[3] 
\textbf{问题:}(闵可夫斯基线性形式定理)设实线性形式为:
\[
L_i(x_1, \dots, x_n) = \sum_{j=1}^n a_{ij} x_j, \quad i = 1, \dots, n,
\]
其中 \(\operatorname{det}(a_{ij}) \neq 0\),且正实数 \(c_1, \dots, c_n\) 满足 \(c_1 \cdots c_n > |\operatorname{det}(a_{ij})|\)。证明存在整数 \(m_1, \dots, m_n \in \mathbb{Z}\),使得:
\[
|L_i(m_1, \dots, m_n)| < c_i, \quad i = 1, \dots, n
\]
\textbf{解答:} 考虑 \(\mathbb{R}^n\) 中的格 \(\Gamma = \mathbb{Z}^n\),\(\operatorname{vol}(\Gamma) = 1\)。考虑子集:
\[
X = \left\{(x_1, \dots, x_n) \in \mathbb{R}^n \mid |L_i(x_1, \dots, x_n)| < c_i, i = 1, \dots, n\right\}
\]
子集 \(X_0 = \prod_{i=1}^n (-c_i, c_i)\) 的体积为 \(2^n c_1 \cdots c_n\),通过变换 \((L_1, \dots, L_n)\) 得到 \(X\),故:
\[
\operatorname{vol}(X) = |\operatorname{det}(\frac{\partial L_i}{\partial x_j})|^{-1} \operatorname{vol}(X_0) = |\operatorname{det}(a_{ij})|^{-1} 2^n c_1 \cdots c_n > 2^n = 2^n \operatorname{vol}(\Gamma)
\]
由闵可夫斯基格点定理,\(X\) 含有一个非零格点 \((m_1, \dots, m_n)\)。

\end{enumerate}

\section{第五章:闵可夫斯基理论}

\subsection{练习题}

\begin{enumerate}

\item[1] 
\textbf{问题:} 写出一个仅依赖于 \(K\) 的常数 \(A\),使得 \(K\) 的每个非零整理想 \(\mathfrak{a}\) 含有一个非零元素 \(a\),满足:
\[
|\tau a| < A (\mathcal{O}_K : \mathfrak{a})^{1/n},
\]
其中 \(n = [K : \mathbb{Q}]\),对所有 \(\tau \in \operatorname{Hom}(K, \mathbb{C})\)。\\

\textbf{解答:} 已知\\
\textbf{定理1.1:} 设 $\mathfrak{a} \neq 0$ 是数域 $K$ 的整理想,且 $c_\tau>0$,其中 $\tau \in \operatorname{Hom}(K, \mathbb{C})$,是满足 $c_\tau=c_{\bar{\tau}}$ 的实数,并且
$$
\prod_\tau c_{\mathfrak{\tau}}>A\left(\mathcal{O}_K: \mathfrak{a}\right)
$$
其中 $A=\left(\frac{2}{\pi}\right)^s \sqrt{\left|d_K\right|}$。则存在 $a \in \mathfrak{a}, a \neq 0$,使得
$$
|\tau a|<c_\tau \quad \text{对所有} \quad \tau \in \operatorname{Hom}(K, \mathbb{C})。
$$

根据定理1.1,
设 \(A = \sqrt[n]{(\frac{2}{\pi})^s \sqrt{|d_K|}}\),则 \(c_\tau = A (\mathcal{O}_K : \mathfrak{a})^{1/n}\) 满足定理 (1.1) 的条件,故存在非零 \(a\),对所有 \(\tau \in \operatorname{Hom}(K, \mathbb{C})\) 满足:
\[
|\tau a| < A (\mathcal{O}_K : \mathfrak{a})^{1/n}
\]

\item[2] 
\textbf{问题:} 证明中心对称的凸集:
\[
X = \left\{(z_\tau) \in K_{\mathbb{R}} \mid \sum_\tau |z_\tau| < t\right\}
\]
的体积为 \(\operatorname{vol}(X) = 2^r \pi^s \frac{t^n}{n!}\)。\\
\textbf{解答:} \(X\) 在 \(\mathbb{R}^{r+2s}\) 中的像为:
\[
f(X) = \left\{(x_\tau) \in \prod_\tau \mathbb{R} \mid \sum_\rho |x_\rho| + 2 \sum_\sigma \sqrt{x_\sigma^2 + x_{\bar{\sigma}}^2} < t\right\}
\]
为简化记号,代 \(x_i\)(\(i = 1, \dots, r\))代替 \(x_\rho\),\(y_j, z_j\)(\(j = 1, \dots, s\))代替 \(x_\sigma, x_{\bar{\sigma}}\)。\(f(X)\) 的体积通过积分计算:
\[
I(t) = \int_{f(X)} dx_1 \cdots dx_r dy_1 \cdots dy_s dz_1 \cdots dz_s
\]
用极坐标 \(y_j = u_j \cos \theta_j, z_j = u_j \sin \theta_j\),得:
\[
I(t) = \int u_1 \cdots u_s dx_1 \cdots dx_r du_1 \cdots du_s d\theta_1 \cdots d\theta_s
\]
积分域为:
\[
\begin{cases}
|x_1| + \cdots + |x_r| + 2u_1 + \cdots + 2u_s < t, \\
0 \leq u_j, \quad j = 1, \dots, s, \\
0 \leq \theta_j \leq 2\pi, \quad j = 1, \dots, s
\end{cases}
\]
代 \(2u_j = w_j\),得:
\[
I(t) = 2^r 4^{-s} (2\pi)^s I_{r,s}(t)
\]
其中:
\[
I_{r,s}(t) = \int w_1 \cdots w_s dx_1 \cdots dx_r dw_1 \cdots dw_s
\]
积分域为:
\[
\begin{cases}
x_1 + \cdots + x_r + w_1 + \cdots + w_s < t, \\
0 \leq x_i, \quad i = 1, \dots, r, \\
0 \leq w_j, \quad j = 1, \dots, s
\end{cases}
\]
显然 \(I_{r,s}(t) = t^n I_{r,s}(1)\)。由傅比尼定理:
\[
\begin{aligned}
I_{r,s}(1) &= \int_0^1 I_{r-1,s}(1 - x_1) dx_1 \\
&= \int_0^1 (1 - x_1)^{n-1} I_{r-1,s}(1) dx_1 \\
&= \frac{1}{n} I_{r-1,s}(1)
\end{aligned}
\]
归纳得:
\[
I_{r,s}(1) = \frac{1}{n(n-1) \cdots (n-r+1)} I_{0,s}(1)
\]
类似地:
\[
I_{0,s}(1) = \int_0^1 w_1 (1 - w_1)^{2s-2} I_{0,s-1}(1) dw_1 = \frac{1}{2s(2s-1)} I_{0,s-1}(1)
\]
故:
\[
I_{0,s}(1) = \frac{1}{(2s)!} I_{0,0}(1) = \frac{1}{(2s)!}
\]
因此 \(I_{r,s}(1) = \frac{1}{n!}\),有:
\[
\operatorname{vol}(X) = 2^s \operatorname{vol}(f(X)) = 2^s \cdot 2^r \cdot 4^{-s} \cdot (2\pi)^s \cdot t^n \cdot I_{r,s}(1) = \frac{2^r \pi^s t^n}{n!}
\]

\item[3] 
\textbf{问题:} 证明在 \(K\) 的整数环 \(\mathcal{O}_K\) 的每个非零理想 \(\mathfrak{a}\) 中,存在非零 \(a\) 满足:
\[
|N_{K \mid \mathbb{Q}}(a)| \leq M (\mathcal{O}_K : \mathfrak{a}),
\]
其中 \(M = \frac{n!}{n^n} \left(\frac{4}{\pi}\right)^s \sqrt{|d_K|}\)(即所谓的闵可夫斯基界)。\\
\textbf{解答:} 考虑紧的、凸的、中心对称集合:
\[
X = \left\{(z_\tau) \in K_{\mathbb{R}} \mid \sum_\tau |z_\tau| \leq n \left(M (\mathcal{O}_K : \mathfrak{a})\right)^{1/n}\right\}
\]
其体积为:
\[
\operatorname{vol}(X) = \frac{2^r \pi^s}{n!} \cdot n^n \cdot \left(\frac{n!}{n^n} \left(\frac{4}{\pi}\right)^s \sqrt{|d_K|}\right) \cdot (\mathcal{O}_K : \mathfrak{a}) = 2^n \operatorname{vol}(\Gamma)
\]
由练习 4.2,\(X\) 含 \(\Gamma = j \mathfrak{a}\) 的非零元素,故存在 \(\mathfrak{a}\) 中的非零 \(a\) 满足:
\[
\begin{aligned}
|N_{K \mid \mathbb{Q}}(a)| &= \prod_\tau |\tau(a)| \\
&\leq \left(\frac{1}{n} \sum_\tau |\tau(a)|\right)^n \\
&\leq M (\mathcal{O}_K : \mathfrak{a})
\end{aligned}
\]

\end{enumerate}


\end{document}
